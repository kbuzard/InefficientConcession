%\documentclass{beamer} 
\documentclass[handout]{beamer} 
\usetheme{Ilmenau}
\usepackage{graphicx,verbatim,hyperref}
\usepackage{textpos}

\usecolortheme{beaver}
\useinnertheme{default}
\setbeamertemplate{itemize item}[triangle]
\setbeamertemplate{itemize subitem}[triangle]
\setbeamertemplate{itemize subsubitem}[circle]
\setbeamertemplate{enumerate items}[default]
\setbeamertemplate{blocks}[upper=block head,rounded]
\setbeamercolor{item}{fg=black}
\usefonttheme{serif} %should allow ccfonts to take effect

\usepackage{cite}
\usepackage{times, verbatim,xcolor,bm}
\usepackage{amsbsy,amssymb, amsmath, amsthm}
\usepackage{booktabs}
%David miller's fonts
	\usepackage[T1]{fontenc}
	\usepackage[boldsans]{ccfonts}
	\usepackage[euler-hat-accent]{eulervm}

\newcommand{\al}{\alpha}
\newcommand{\expect}{\mathbb{E}}
\newcommand{\Bt}{B(\bm{\tau^a})}
\newcommand{\bta}{\bm{\tau^a}}
\newcommand{\btn}{\bm{\tau^{tw}}}
\newcommand{\ga}{\gamma}
\newcommand{\ve}{\varepsilon}
\newcommand{\ta}{\theta}
\newcommand{\de}{\delta}
\newcommand{\ov}{\overline}

\newenvironment{changemargin}[2]{% 
  \begin{list}{}{% 
    \setlength{\topsep}{0pt}% 
    \setlength{\leftmargin}{#1}% 
    \setlength{\rightmargin}{#2}% 
    \setlength{\listparindent}{\parindent}% 
    \setlength{\itemindent}{\parindent}% 
    \setlength{\parsep}{\parskip}% 
  }% 
  \item[]}{\end{list}} 

\begin{document}
\title[Inefficient Concessions and Mediation\hspace{2.95in}\insertframenumber/\inserttotalframenumber]{Inefficient Concessions and Mediation}
\author[Kristy Buzard and Ben Horne]{Kristy Buzard and Ben Horne\\ kbuzard@syr.edu}
\date{December 3, 2016}
\maketitle
%\insertpresentationendpage removed b/c of appendix



\section{Overview}
\subsection{Preview}
\begin{frame}
\frametitle{Unrecognized States}
Unrecognized states (URS): control and govern territory, seek recognition
\pause
\begin{itemize}[<+->]
  \item Six current URS more than 20 yrs old (+ Eastern Ukraine)
	\item No explanation in the literature for how this can be a stable outcome
\end{itemize}
\end{frame}


\begin{frame}{What we do}

\pause
We demonstrate unrecognized statehood can be a ``status quo'' outcome
\pause
\begin{itemize}[<+->]
	\item (SPN) Equilibrium in a repeated game
	\item Four players
		\begin{itemize}[<+->]
			\item Home government
			\item Secessionist elite
			\item Patron state
			\item International community
		\end{itemize}
	\item State variable: Status Quo (SQ) payoffs for secessionists
\end{itemize}

\end{frame}



\begin{frame}{The General Idea}

\pause
In each period: secessionists and gov't each choose (simultaneously) among $\left\{\text{Fight, Status Quo, Cede}\right\}$
\pause
		\begin{itemize}
			\item We need Status Quo to be a stage game best response for both secessionists and gov't
		\end{itemize} 

\pause
\vskip.2in
We add some more realistic elements:
\pause
		\begin{enumerate}[<+->]
			\item Unrecognized status reduces Status Quo payoffs of secessionists each period
			\item Patron and int'l community can make investments in both actors' payoffs
		\end{enumerate}

\end{frame}


\begin{frame}{Outline of Talk}
\pause
\begin{enumerate}[<+->]
	\item M
	\item S
	\item W
\end{enumerate}
\end{frame}



\section{Model}
\subsection{Economic and Political Structure}
\begin{frame}{Timeline}
\pause
\end{frame}


\subsection{The Players}

\begin{frame}{Home State Actors}
	\pause
Central Government of Home State ($g$): desires reunification \\
\pause
Secessionists ($s$): desire recognized independence
\pause
\begin{itemize}	
	\item Central issue of contention: recognized independence vs. reunification.
\end{itemize}

\pause
\vskip.2in
Assumptions:
\begin{itemize}[<+->]
	\item Issue of status is indivisible, highly valued by both sides
	\item Insufficient credible side payments for easy settlement
\end{itemize}
\end{frame}


\begin{frame}{Stage Game between Home Gov't $\&$ Secessionists}
\pause

\begin{center}
\begin{tabular}{|l|c|c|}
  \hline      & Trust & Fight \\ \hline
	 Trust& T, T& -D, T+W \\ \hline
	Fight & T+W,-D& W-D, W-D \\ \hline
\end{tabular}
\end{center}

\vskip.05in
where 
\begin{itemize}
	\item $T\geq0$: Benefit from the other country playing Trust
	\item $W\geq0$: Additional benefit from playing Fight
	\item $D\geq0$: Damages due to the other country playing Fight
\end{itemize}
\vskip.05in
\pause
Assume $T>W-D$
\end{frame}

\begin{frame}
\begin{itemize}[<+->]
	\item Payoffs: sum the discounted stage game payoffs plus any concessions 
		\begin{itemize}
			\item e.g. player's $i$'s payoff if both parties play ``Trust'' in every period: $\sum_{t=1}^\infty \de_i^{t-1} T = \frac{T}{1-\de_i}$
		\end{itemize}
	\item Parameters are common knowledge with the exception of $\delta_i$, which is country i's private information
	\item Social welfare measured as sum of high types' expected utilities
\end{itemize}
\end{frame}


\section{Benchmark}
\subsection{No Money Burning}


\begin{frame}{Benchmark Model}
\pause
Assume two types: $\de_h$ and $\de_l$
\begin{itemize}[<+->]
	\item $\de_h > \de^* > \de_l$ where $\de^*$ is the cutoff for sustaining (Trust,Trust) eqm
	\item value of any concession to recipient = cost to giver
\end{itemize}

\pause
\vskip.2in
Some equilibria of interest
\begin{itemize}[<+->]
	\item Pool on `Fight'
	\item Separating without concessions
	\item Separating through concessions
\end{itemize}
\end{frame}

\begin{frame}{Separating through concessions}
\pause
\begin{beamerboxesrounded}[upper=palette tertiary, shadow=true]{Theorem 2}
  In the best concessions separating equilibrium, high types give the smallest concession necessary to separate. Low types do not give a concession.
\end{beamerboxesrounded}
	
	\pause
	\vskip.2in
	\begin{itemize}
		\item The smallest concession is $p(T+D)$
		\item If $p$ is low, high types are better off in the `fight' pooling equilibrium
	\end{itemize}
\end{frame}

\section{Inefficient Concessions}
\subsection{Add Money Burning}
\begin{frame}{S}
Assume cost of concession is $g$, the concession itself
\begin{itemize}
  \item Allow giver of concession to also choose $0 \leq e \leq 1$, where benefit of concession is $eg$
\end{itemize}


\begin{beamerboxesrounded}[upper=palette tertiary, shadow=true]{P}
  With the 

\end{beamerboxesrounded}

\end{frame}

\begin{frame}{S}
\begin{beamerboxesrounded}[upper=palette tertiary, shadow=true]{P}
  A
\end{beamerboxesrounded}

\end{frame}


\section{Conclusion}
\subsection{}
\begin{frame}{Conclusion}
\begin{itemize}[<+->]
	\item We present 
\end{itemize}

\end{frame}


\end{document}