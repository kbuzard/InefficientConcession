\documentclass[12pt]{article}

\addtolength{\textwidth}{1.4in}
\addtolength{\oddsidemargin}{-.7in} %left margin
\addtolength{\evensidemargin}{-.7in}
\setlength{\textheight}{8.5in}
\setlength{\topmargin}{0.0in}
\setlength{\headsep}{0.0in}
\setlength{\headheight}{0.0in}
\setlength{\footskip}{.5in}
\renewcommand{\baselinestretch}{1.0}
\setlength{\parindent}{0pt}
\linespread{1.1}

\usepackage[pdftex,
bookmarks=true,
bookmarksnumbered=false,
pdfview=fitH,
bookmarksopen=true]{hyperref}

\usepackage{amssymb, amsmath, amsthm, bm}
\usepackage{graphicx,csquotes,verbatim}
\usepackage[backend=biber,block=space,style=authoryear]{biblatex}
\setlength{\bibitemsep}{\baselineskip}
\usepackage[american]{babel}
%dell laptop
\addbibresource{C:/Users/Kristy/Dropbox/Research/xBibs/tradeagreements.bib}
%\addbibresource{C:/Users/Kristy/Documents/Dropbox/Research/xBibs/tradeagreements.bib}
\renewcommand{\newunitpunct}{,}
\renewbibmacro{in:}{}


\DeclareMathOperator*{\argmax}{arg\,max}
\usepackage{xcolor}
\hbadness=10000

\newtheorem{proposition}{Proposition}
\newcommand{\ve}{\varepsilon}
\newcommand{\ov}{\overline}
\newcommand{\un}{\underline}
\newcommand{\ta}{\theta}
\newcommand{\al}{\alpha}
\newcommand{\Ta}{\Theta}
\newcommand{\expect}{\mathbb{E}}
\newcommand{\Bt}{B(\bm{\tau^a})}
\newcommand{\bta}{\bm{\tau^a}}
\newcommand{\btn}{\bm{\tau^n}}
\newcommand{\btw}{\bm{\tau^{tw}}}
\newcommand{\ga}{\gamma}
\newcommand{\Ga}{\Gamma}
\newcommand{\de}{\delta}

\begin{document}
\begin{center}
  Working notes for Inefficient Concessions
\end{center}

\vskip.3in
\un{June 6, 2017}\\
Want to develop result showing that mediator increases potential for peace
\begin{itemize}
	\item i.e. want to show mediator makes (Trust, Trust) and equilibrium over a larger parameter space than without mediator
\end{itemize}

\vskip.2in
What do we know?
\begin{itemize}
	\item Theorem 4: Under some parameters, optimal concessions aren't made when there isn't trust
		\begin{itemize}
			\item This is still a separating equilibrium, but completely inefficient concessions are given. So welfare is lower because of inefficient concessions. 
			\item Where there is a possibility of mediation helping to achieve peace where it otherwise would not be attainable (not just improving welfare) is if this reduction in welfare due to the inefficient concessions means that for some parameters it's not worth separating so we don't get peace at all without the mediator.
				\begin{itemize}
					\item i.e. when welfare for high type from separating falls below that for pooling, separating is no longer an equilibrium (convo with Jean-Guillaume).
				\end{itemize}
		\end{itemize}
	\item Theorem 5: Mediator eliminates inefficient concessions
		\begin{itemize}
			\item Here, we're already in that parameter space where there \emph{are} inefficient concessions
		\end{itemize}
\end{itemize}

\vskip.5in
Will need to start by assuming that $\de_h <$ threshold where separating with no concessions works.

\vskip1in
Intermediate step for Theorem 4
\begin{itemize}
	\item Low type IC constraint to solve for equilibrium high type gift
	\item Already having shown $g_l = 0$, can set $g_h = g$ for simplicity of notation
	\item Also, since all the discount factors are for the low type, I'll let $\de_l = \de$
	\item For the case where $e=1$:

\end{itemize}
IC constraint:
\[
  X_{FF}^l \geq p X_{FT}^l + (1-p) X_{FF}^l - g
\]
Expanded in basic terms (without efficiency issues)
\[
  \frac{W-D}{1-\de} \geq p \left[T + W + \frac{\de}{1-\de}\left(W-D\right)\right] + (1-p) \frac{W-D}{1-\de} - g
\]
Multiply through by $\left(1-\de \right)$
\[
  W-D \geq p\left(1-\de \right) \left[T + W\right] + p\de \left(W-D\right) + (1-p) \left[W-D\right] - \left(1-\de \right) g
\]
Now add complexity from Table 2
\begin{multline*}
  W(1+(1-\al_1)g_2)-D(1+(1-\al_2)g_1) \geq \\
	p\left(1-\de \right) \left[T(1+\al_2 g_1) + W(1+(1-\al_1)g_2)\right] + p\de \left(W(1+(1-\al_1)g_2)-D(1+(1-\al_2)g_1)\right) \\ + (1-p) \left[W(1+(1-\al_1)g_2)-D(1+(1-\al_2)g_1)\right] - \left(1-\de \right) g
\end{multline*}
Substitute in $\al_1 = 0$ everywhere since this is the IC for a low-type of player 1:
\begin{multline*}
  W(1+g_2) - D(1+(1-\al_2)g_1) \geq \\
	p\left(1-\de \right) \left[T(1+\al_2 g_1) + W(1+g_2) \right] + p\de \left(W(1+g_2) - D(1+(1-\al_2)g_1)\right) \\ + (1-p) \left[W(1+g_2) - D(1+(1-\al_2)g_1)\right] - \left(1-\de \right) g
\end{multline*}
Set $g_1 = 0$ on the LHS and $g_1 = g$ on the RHS:
\begin{multline*}
  W(1+g_2) - D \geq \\
	p\left(1-\de \right) \left[T(1+\al_2 g) + W(1+g_2) \right] + p\de \left(W(1+g_2) - D(1+(1-\al_2)g)\right) \\ + (1-p) \left[W(1+g_2) - D(1+(1-\al_2)g)\right] - \left(1-\de \right) g
\end{multline*}
Set $g_2 = 0$ and $\al_2 = 0$ wherever there is a $(1-p)$ and $g_2 = g$ and $\al_2 = 1$ wherever there is a $p$:
\begin{multline*}
  pW(1+g) + (1-p)W - D \geq \\
	p\left(1-\de \right) \left[T(1+g) + W(1+g) \right] + p\de \left(W(1+g) - D\right) \\ + (1-p) \left[W - D(1+g)\right] - \left(1-\de \right) g
\end{multline*}
Expand
\begin{multline*}
  pW + pWg + W - pW - D \geq \\
	\left(p-p\de \right) \left[T+Tg + W+Wg \right] + p\de \left(W+Wg - D\right) \\ + (1-p) \left[W - D -Dg\right] - \left(1-\de \right) g
\end{multline*}
Cancel some like terms and move $\left(1-\de \right) g$ to LHS
\begin{multline*}
  \left(1-\de \right) g + pWg + W - D \geq \\
	p\left[T+Tg + W+Wg \right] -p\de \left[T+Tg \right] - p\de D \\ + (1-p) \left[W - D -Dg\right] 
\end{multline*}
Do some more canceling and expanding
\begin{multline*}
  \left(1-\de \right) g + W - D \geq \\
	p\left[T+Tg + W\right] -p\de T -p\de Tg - p\de D \\ + W - D -Dg -pW +pD +pDg
\end{multline*}
\[
  \left(1-\de \right) g \geq 
	pT+pTg + pW -p\de T -p\de Tg - p\de D  -Dg -pW +pD +pDg
\]
\[
  \left(1-\de \right) g \geq pT+pTg -p\de T -p\de Tg - p\de D  -Dg +pD +pDg
\]
Now just rearrange to get all the $g$ terms on the left
\[
  \left(1-\de \right) g -pTg + p\de Tg + Dg - pDg\geq pT -p\de T - p\de D +pD
\]
\[
  \left[\left(1-\de \right) -p(1-\de)T +(1-p)D\right]g \geq p(1-\de)(T + D)
\]
\[
  g \geq \frac{p(1-\de)(T + D)}{\left(1-\de \right) -p(1-\de)T +(1-p)D}
\]
In comparing to the minimum separating gift when $e=0$, which is
\[
  g^* \geq \frac{p(1-\de)(T + D)}{\left(1-\de \right)}
\]
We can simplify to see that the minimum gift when $e=1$ is larger when
\[
  (1-\de)pT > (1-p)D
\]

\vskip.5in
Next we get the threshold for $\de_h$ that is necessary for a concessions separating eqm to exist:
\begin{itemize}
	\item Use the high-type IC constraint:
		\[
		  p X^h_{TT} + (1-p) X^h_{FF} - g_h \geq X^h_{FF}
		\]
	\item We'll need to expand for material effects but then solve for both $e=0$ and $e=1$
	\item For $e=1$:
		\[
		  \frac{p}{1-\de_h}T(1+g) + \frac{1-p}{1-\de_h}(W - D(1+g)) - g_h \geq \frac{1}{1-\de_h}(W-D)
		\]
		\[
		  pT + pTg + (1-p)(W - D - Dg) - g_h(1-\de_h) \geq W-D
		\]
		\[
		  pT + pTg + W - D - Dg - pW + pD + pDg - g_h + g_h \de_h \geq W-D
		\]
		\[
		  pT + pTg - Dg - pW + pD + pDg - g_h + g_h \de_h \geq 0
		\]
		\[
		  pT + pTg - Dg - pW + pD + pDg - g_h + g_h \de_h \geq 0
		\]
\end{itemize}

\newpage
To-do list after Venice CesIfo 06/15/17:
\begin{itemize}
	\item Add discussion of intuitive criterion to text (if I end up needing it): out of eqm beliefs put zero weight on types that can never gain from deviating from a fixed eqm outcome
	\item Pin down whether it matters for mediator outcome if $c_h = c$ or $c_h = c(g)$
	\item Incorporate mediator result into text; re-organize text so it matches June presentation more closely
	\item Note for mediator result:
		\begin{itemize}
			\item For two high types to cooperate in the mechanism, need large enough gift to screen out low types. They're still just reporting, so don't give a gift if not matched with another high type. PLUS when they're exchanging gifts, the gifts will actually be used for good:
				\[
				  T(1+g) \geq (1-\de)[T(1+g) + W] + \de (W-D)
				\]
				\[
				  \de \geq \frac{W}{T + Tg + D}
				\]
				Note that this threshold is smaller than $\frac{W}{T + D}$, the one we get without concessions having material benefit / harm
		\end{itemize}
	\item Jim Fearon: maybe get rid of repeated game and just parameterize a one-shot game; he has a paper about concessions being used against you; he also gave me the reference for another one but I've forgotten it
	\item Someone who \textit{can't} cooperate under either $e=0$ or $e=1$ CAN cooperate under mediator
		\begin{itemize}
			\item In the absence of mediation, being able to burn money can help, but you still lose something
			\item I still have a question about the overall game: if $e=0$ provides higher welfare if it were enforceable but it's not, do we just go with $e=1$? This question goes away if I get rid of the repeated game.
			\item Some parameters. For all of these, I take $W=8$, $D=5$, $T=10$
				\begin{itemize}
					\item $p=.4, \de_l = .45, \de_h = \de_{STC} = .53$. Choose $e=1$, but $\de_1 = .65$
					\item $p=.3, \de_l = .45, \de_h = \de_{STC}  = .53$. Choose $e=1$, but $\de_1 > 1$; even worse at $p=.2$
					\item $p=.4, \de_l = .5, \de_h = \de_{STC} = .53$. Choose $e=1$, $\de_1 \geq .4$ (works fine)
					\item $p=.1, \de_l = .1, \de_h = \de_{STC} = .53$. Choose $e=0$, but $\de_0 \geq 6.7$
					\item $p=.05, \de_l = .5, \de_h = .7$. Choose $e=0$, but $\de_0 = 7.6$
					\item $p=.1, \de_l = .5, \de_h = .75$. Choose $e=0$, but $\de_0 = 6.7$ and $\de_1 \geq 3$
				\end{itemize}
		\end{itemize}
\end{itemize}
\end{document}