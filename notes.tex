\documentclass[12pt]{article}

\addtolength{\textwidth}{1.4in}
\addtolength{\oddsidemargin}{-.7in} %left margin
\addtolength{\evensidemargin}{-.7in}
\setlength{\textheight}{8.5in}
\setlength{\topmargin}{0.0in}
\setlength{\headsep}{0.0in}
\setlength{\headheight}{0.0in}
\setlength{\footskip}{.5in}
\renewcommand{\baselinestretch}{1.0}
\setlength{\parindent}{0pt}
\linespread{1.1}

\usepackage[pdftex,
bookmarks=true,
bookmarksnumbered=false,
pdfview=fitH,
bookmarksopen=true]{hyperref}

\usepackage{amssymb, amsmath, amsthm, bm}
\usepackage{graphicx,csquotes,verbatim}
\usepackage[backend=biber,block=space,style=authoryear]{biblatex}
\setlength{\bibitemsep}{\baselineskip}
\usepackage[american]{babel}
%dell laptop
\addbibresource{C:/Users/Kristy/Dropbox/Research/xBibs/tradeagreements.bib}
%\addbibresource{C:/Users/Kristy/Documents/Dropbox/Research/xBibs/tradeagreements.bib}
\renewcommand{\newunitpunct}{,}
\renewbibmacro{in:}{}


\DeclareMathOperator*{\argmax}{arg\,max}
\usepackage[dvipsnames]{xcolor}
\hbadness=10000

\newtheorem{proposition}{Proposition}
\newcommand{\ve}{\varepsilon}
\newcommand{\ov}{\overline}
\newcommand{\un}{\underline}
\newcommand{\ta}{\theta}
\newcommand{\al}{\alpha}
\newcommand{\Ta}{\Theta}
\newcommand{\expect}{\mathbb{E}}
\newcommand{\Bt}{B(\bm{\tau^a})}
\newcommand{\bta}{\bm{\tau^a}}
\newcommand{\btn}{\bm{\tau^n}}
\newcommand{\btw}{\bm{\tau^{tw}}}
\newcommand{\ga}{\gamma}
\newcommand{\Ga}{\Gamma}
\newcommand{\de}{\delta}

\begin{document}
\begin{center}
  Working notes for Inefficient Concessions
\end{center}

\vskip.3in

\section{Alternative Examples}

\subsection{Destroying weapons}
I'm not sure this is perfect: it's a bit of a stretch to say the weapons could be given to the negotiating partner who could use them to build up civil society, but I guess if they're given to non-militarized police to enforce law and order...
\begin{itemize}
	\item On disarmament in general: https://cain.ulster.ac.uk/events/peace/docs/disarmament.htm
	\item FARC in 2017 (Colombia)
		\begin{itemize}
			\item https://www.crisisgroup.org/latin-america-caribbean/andes/colombia/statement-transitional-justice-agreement-between-colombian-government-and-farc
			\item https://news.un.org/en/story/2017/06/560442-colombia-un-mission-collects-nearly-all-remaining-weapons-farc-ep
			\item https://unmc.unmissions.org/en/un-mission-finalises-extraction-arms-caches-and-laying-down-weapons-farc-ep
		\end{itemize}
	\item Northern Ireland
		\begin{itemize}
			\item Report of the Independent International Commission on Decomissioning: https://cain.ulster.ac.uk/events/peace/decommission/iicd260905.pdf
			\item Note two years later: https://www.crisisgroup.org/crisiswatch/october-2007
			\item Summary: https://cain.ulster.ac.uk/events/peace/decommission.htm
		\end{itemize}
		\item Philippines / Bangsamoro region: https://www.crisisgroup.org/asia/south-east-asia/philippines/301-philippines-militancy-and-new-bangsamoro
	\item Yemen, \url{https://www.crisisgroup.org/crisiswatch/print?page=1&location%255B%255D=28&date_range=custom&from_month=01&from_year=2020&to_month=04&to_year=2020&t=CrisisWatch+Database+Filter}
		\begin{itemize}
			\item https://www.arabnews.com/node/1292091/middle-east
			\item As part of “Phase 2” agreement...Rival Yemeni groups 24 Jan completed transfer of heavy weapons in Aden to Saudi-controlled base in city within fifteen-day timeframe. After early redeployments from Aden, process again stalled late Jan, with local appointments still not made. 
		\end{itemize}
	\item Liberia: https://peaceaccords.nd.edu/accord/accra-peace-agreement, https://www.un.org/press/en/1999/19990726.SGSM7078.html
\end{itemize}

\subsection{Other conflicts}
I investigated these for different types of inefficiencies but did not find anything useful
\begin{itemize}	
	\item North and South Korea
	\item What happened to make peace in Sudan?
	\item Liberia--don't see anything in the Accra accords
	\item Taliban in Afghanistan? February 29, 2020
		\begin{itemize}
			\item No help from International Crisis Group
			\item Read agreement on state.gov; it's only four pages long, not much too it
		\end{itemize}
	\item Iran deal with the U.S.: JCPOA
\end{itemize}

\section{No material value CSE vs. repeated game}
\subsection{Patience threshold}
Minimum $\de$ for $e=0$ CSE / no material value case to be sustainable. Remember that $g^0 = p(T+D)$
						\[
						  \frac{1}{1-\de_h}\left\{pT + (1-p)(W-D)\right\} - (1-p)g_0 \geq \frac{1}{1-\de_h} (W-D)
						\]
						\[
						  pT + (1-p)(W-D) - (1-\de_h)(1-p) g^0 \geq W-D
						\]
						\[
						  \de_h (1-p) g^0 \geq W-D - pT - (1-p)(W-D) +g^0(1-p)
						\]
						\[
						  \de_h (1-p) g^0 \geq p(W-D-T) +g^0(1-p)
						\]
						\[
						  \de_h \geq \frac{p(W-D-T)}{g^0(1-p)} + 1 = \frac{p(W-D-T)}{p(T+D)(1-p)} + 1 
						\]
						\[
							\de_h \geq \frac{W}{(T+D)(1-p)} + \frac{(-D-T)}{(T+D)(1-p)} + 1 = \frac{W}{(T+D)(1-p)} - \frac{1}{1-p} + 1
						\]
						\[
							\de_h \geq \frac{W}{(T+D)(1-p)} - \frac{p}{1-p}
						\]
						The following shows that this is strictly smaller than $\de^h$. \textbf{This means that the set of types who can make and sustain peace under the CSE with no material value / CSE with material value and $e=0$ is larger than the set of types who can make and sustain peace under the repeated game.}
							\begin{itemize}
								\item $T+D>W$
								\item $-(T+D)<-W$
								\item $-p(T+D)<-pW$
								\item $W-p(T+D)<W-pW$
								\item $\frac{W-p(T+D)}{T+D}<\frac{W-pW}{T+D} = \frac{(1-p)W}{T+D}$
								\item $\frac{W-p(T+D)}{(1-p)(T+D)} = \frac{W}{(1-p)(T+D)} - \frac{p(T+D)}{(1-p)(T+D)} = \frac{W}{(1-p)(T+D)} - \frac{p}{1-p}< \frac{W}{T+D}$. Q.E.D.
							\end{itemize}
							
\subsection{Welfare}
Welfare under no material value CSE is always at least as large as welfare under repeated-game grim trigger
\[
  p\frac{T}{1-\de_h} + (1-p)\frac{W-D}{1-\de_h} - p(T+D) \geq p\frac{T}{1-\de_h} + (1-p)\frac{-D}{1-\de_h} + (1-p)\frac{\de_h}{1-\de_h}(W-D)
\]
\[
  (1-p)\frac{W-D}{1-\de_h} - p(T+D) \geq (1-p)\frac{-D}{1-\de_h} + (1-p)\frac{\de_h}{1-\de_h}(W-D)
\]
\[
  (1-p)(W-D) - p(1-\de_h)(T+D) \geq -(1-p)D + (1-p)\de_h(W-D)
\]
\[
  (1-p)W - p(1-\de_h)(T+D) \geq (1-p)\de_h(W-D)
\]
\[
  (1-p)W -(1-p)\de_h(W-D)\geq  p(1-\de_h)(T+D) 
\]
\[
  W-pW - \de_hW+\de_hD +p\de_hW-p\de_hD\geq  pT + pD -p\de_hT - p\de_hD 
\]
\[
  W-pW - \de_hW+\de_hD +p\de_hW \geq  pT + pD -p\de_hT 
\]
\[
  W - \de_hW+\de_hD +p\de_hW \geq  pT + pD + pW -p\de_hT 
\]
Because $T +D > W$, we have 
\[
  W - \de_hW+\de_hD +p\de_hW \geq  pT + pD + pW -p\de_hT > -p\de_hT 
\]
\[
  (1-\de_h)W +\de_hD +p\de_hW + p\de_hT > 0 
\]
Each term on the left is strictly larger than 0, so this statement is always true.
\begin{itemize}
	\item The above calculations are not quite right: the LHS subtracts off the full value of the gift, but I forgot to include that you also receive the gift $p$ of the time. So the LHS is strictly larger than what I have here, which means this result is even a bit stronger.
\end{itemize}

\section{Make mediator work}
November 29, 2020: In making revisions in April 2020 to respond to the EJPE referee reports, I discovered the IC constraint does not hold for the low type in Section 5: Mediation. Here I will see what adjustments might make the mediator work.
\begin{itemize}
	\item November 30: I showed formally and in great detail in Section \ref{sec:e1} that the IC constraint does not bind for $e=1$ with our standard assumptions.
	\item December 1: I began working on the $e < 1$ case in Section \ref{sec:eless}
\end{itemize}

\subsection{Mediation: Basic Details and Overview}
\begin{itemize}
	\item If the mediator gets two `high' reports, both countries are instructed to exchange gifts and play `Trust'
		\begin{itemize}
			\item If the mediator gets one or zero `high' reports, there is no gift and they both play `Fight'
		\end{itemize}
	\item I show the constraints for both types in Section \ref{sec:const}.
	\item I detail the choices of all types in regards to investments in civil society in Section \ref{sec:civ}.
	\item I show in Section \ref{sec:e1} that the low type's IC constraint cannot hold if $e=1$.
	\item Section \ref{sec:eless} develops the model with $0<e<1$.
\end{itemize}


\subsection{IC constraints for Mediation}
\label{sec:const}
\begin{itemize}
	\item High type IC:
		$$pX_{TT}^h+(1-p)X_{FF}^h - pg_h +peg_h \geq X_{FF}^h $$
			\begin{itemize}
				\item Relative to Section 4.2 in the draft, $-g_h$ there becomes $-pg_h$ here (again, only give gift if matched with high type). Also $+peg_h$ in Section 4.2 disappears on RHS.
			\end{itemize}
	\item Low type IC:
		$$X_{FF}^l \geq pX_{FT}^l+(1-p)X_{FF}^l-pg_h+peg_h $$
			\begin{itemize}
				\item If a low type reveals truthfully: you get $X_{FF}$ and no gift with probability $p$; you get the same with probability $(1-p)$ (you need both countries to be high types in order to do anything)
				\item If a low type lies: get $X_{FT}$ and give gift ($g_h$) and receive gift ($eg_h$) with probability $p$; $X_{FF}$ again with probability $(1-p)$
				\item Relative to Section 4.2, there is no $peg_h$ on the LHS (in Section 4.2, it cancels out with the $peg_h$ on the RHS): no benefit of getting the gift from the high type. Also, $-g_h$ on the RHS becomes $-pg_h$ since you only send the gift if matched with a high type
			\end{itemize}
\end{itemize}

\subsection{Civil society investment choices}
\label{sec:civ}
In the mechanism, the high type has the incentive to invest all of the gift in civil society if it reports truthfully. A high country doesn't make any investment when it reports as a low type because no gift will be specified.
	\begin{itemize}
		\item If there is an exchange of gifts (whether in the mechanism or a truthfully revealed separating equilibrium), you know you're both receiving from and giving to a high type
			\begin{itemize}
				\item \textbf{I want to know whether it depends on the linear form, but that is beyond the scope of the paper at this point}
			\end{itemize}
	\end{itemize}
$\al_i$ is the proportion of the gift that country $i$ receives that it invests in civil society.
	\begin{itemize}
		\item $\al_i = 1$ means the whole value of the gift is invested in civil society. High types make this choice
			\begin{itemize}
				\item $\al_i = 0$ means the whole value of the gift is invested in military buildup
			\end{itemize}
			
		\item High type of Country 1 chooses $\al_1=1$ (all into civil society) 
			\begin{itemize}
				\item $(1-\al_1) = 0$ (nothing into military)
			\end{itemize}
		\item Low type of Country 1 chooses $\al_1=0$ (nothing into civil society) 
			\begin{itemize}
				\item $(1-\al_1) = 1$ (all into military)
			\end{itemize}
		\item High type of Country 2 chooses $\al_2=1$ (all into civil society) 
			\begin{itemize}
				\item $(1-\al_2) = 0$ (nothing into military)
			\end{itemize}
		\item Low type of Country 2 chooses $\al_2=0$ (nothing into civil society) 
			\begin{itemize}
				\item $(1-\al_2) = 1$ (all into military)
			\end{itemize}	
	\end{itemize}



\subsection{Why does low type IC not hold with efficient gifts?}
\label{sec:e1}
\begin{equation}
	X_{FF}^l \geq pX_{FT}^l+(1-p)X_{FF}^l-pg_h+peg_h
	\label{eqn:IClow}
\end{equation}
With our restrictions (mostly the linear assumption, IIRC), High type gives gift with efficiency $e=1$, so gift giving and receiving cancels out. This leaves
		$$X_{FF}^l \geq pX_{FT}^l + X_{FF}^l - pX_{FF}^l$$
		which simplifies to 
		$$X_{FF}^l \geq X_{FT}^l.$$
		Expanding, we have
		\[
			\frac{W\left( \right)-D\left( \right)}{1-\de^l} \geq T\left( \right) + W\left( \right) + \frac{\de^l}{1-\de^l}\left(W\left( \right) -D\left( \right) \right)
		\]
		Now we have to fill in all the complicated bits in parentheses from Table 3 in Section 4.2
		\begin{multline}
			W\left(m_1 + \left(1-\al_1\right)eg_2\right)-D\left(m_2 + \left(1-\al_2\right)eg_1\right) \geq \\
			\left(1-\de^l\right) \left[T\left(s_2 + \al_2eg_1\right) + W\left(m_1 + \left(1-\al_1\right)eg_2\right) \right] \\
			+ \de^l \left(W\left(m_1 + \left(1-\al_1\right)eg_2\right)-D\left(m_2 + \left(1-\al_2\right)eg_1\right)\right)
			\label{eq:lowICex}
		\end{multline}

We've already assumed $e=1$ here as given in Lemma 2 for a concessions separating equilibrium 
	\begin{itemize}
		\item We \textit{should} assume $s_1,s_2,m_1,m_2$ to all be 1 to be consistent with the theory though. If we implement all that AND $e=1$, we get
			\begin{multline*}
					W\left(1 + \left(1-\al_1\right)g_2\right)-D\left(1 + \left(1-\al_2\right)g_1\right) \geq \left(1-\de^l\right) \left[T\left(1 + \al_2g_1\right) + W\left(1 + \left(1-\al_1\right)g_2\right) \right] \\
						+ \de^l \left(W\left(1 + \left(1-\al_1\right)g_2\right)-D\left(1 + \left(1-\al_2\right)g_1\right)\right)
			\end{multline*}
		\item We also know that there are no gifts exchanged for $X^l_{FF}$, so $g_1 = g_2 = 0$ on the LHS. On the RHS, gifts ARE exchanged because player 2 is a high type and the low type of player 1 lied. So we can set $g_1 = g_2 = g_M$
			\begin{multline*}
					W - D \geq \left(1-\de^l\right) \left[ T\left(1 + \al_2g_M\right) + W\left(1 + \left(1-\al_1\right)g_M\right) \right] \\
					+ \de^l \left(W\left(1 + \left(1-\al_1\right)g_M\right)-D\left(1 + \left(1-\al_2\right)g_M\right)\right)
			\end{multline*}
		\item The $\al_i$'s. Recall $\al_i$ represents the portion of the received gift that Player $i$ invests in civil society.
			\begin{itemize}
				\item Player 1 is low type (it is the low type IC constraint that I'm analyzing) and so $\al_1 = 0$.
				\item Player 2 is a high type in what's left of the RHS, so $\al_2 = 1$.
					\begin{itemize}
						\item The $(1-p)$ part of the RHS was combined with the LHS; it was for the low type of Player 2, so it matched with the LHS where there are no gifts since the mediator instructs no gifts if one low-type player tells the truth, which Player 2 does when we're looking at Player 1's IC constraint.
					\end{itemize}
				\item With $a_1 = 0$ and $a_2 = 1$, we have
					\begin{equation}
							W - D \geq \left(1-\de^l\right) \left[ T\left(1 + g_M\right) + W\left(1 + g_M\right) \right] + \de^l \left(W\left(1 + g_M\right)-D\right)
							\label{eqn:alphas}
					\end{equation}
			\end{itemize}
				
		\item Distributing, we have 
			\[
					W - D \geq T + Tg_M + W + Wg_M - \de^l T - \de^l Tg_M - \de^l W - \de^l Wg_M + \de^l W + \de^l Wg_M - \de^l D
			\]
				\item Canceling terms:
			\[
					- D \geq T + Tg_M + Wg_M - \de^l T - \de^l Tg_M - \de^l D
			\]
		\item Rearranging:
			\[
					0 \geq \left(1 - \de^l \right) T + \left(1 - \de^l \right) Tg_M + Wg_M + \left(1 - \de^l \right) D
			\]
			Assuming $g_M > 0$, this inequality can never hold since each of $T,W,D\geq0$ and $0 \leq \left(1 - \de^l \right) \leq 1$			
		\end{itemize}


\subsection{What if gifts are inefficient?}
\label{sec:eless}
Here I attempt to use inefficient gifts to get the low type to truthfully reveal.\\

Equation \ref{eqn:IClow} is the general form of the IC constraint. In the next step in Section~\ref{sec:e1}, I let the gifts cancel out because we assume in that section that $e=1$. I need to go back to Equation \ref{eqn:IClow} here since $e$ is not assumed to be 1.
\[
	X_{FF}^l \geq pX_{FT}^l+(1-p)X_{FF}^l-pg_h+peg_h
\]
Leaving the two gift terms on the end intact, I simplify in the same way as in Section $\ref{sec:e1}$ to arrive at the correct analog to Equation~\ref{eq:lowICex}.
		\begin{multline*}
			W\left(m_1 + \left(1-\al_1\right)eg_2\right)-D\left(m_2 + \left(1-\al_2\right)eg_1\right) \geq \\
			\left(1-\de^l\right) \left[T\left(s_2 + \al_2eg_1\right) + W\left(m_1 + \left(1-\al_1\right)eg_2\right) \right] \\
			+ \de^l \left(W\left(m_1 + \left(1-\al_1\right)eg_2\right)-D\left(m_2 + \left(1-\al_2\right)eg_1\right)\right) -pg_h+peg_h
		\end{multline*}
	\begin{itemize}
		\item We assume $s_1,s_2,m_1,m_2$ to all be 1 to be consistent with Section 4.2
			\begin{multline*}
				W\left(1 + \left(1-\al_1\right)eg_2\right)-D\left(1 + \left(1-\al_2\right)eg_1\right) \geq \\
			\left(1-\de^l\right) \left[T\left(1 + \al_2eg_1\right) + W\left(1 + \left(1-\al_1\right)eg_2\right) \right] \\
			+ \de^l \left(W\left(1 + \left(1-\al_1\right)eg_2\right)-D\left(1 + \left(1-\al_2\right)eg_1\right)\right) -pg_h+peg_h
			\end{multline*}
		\item We know that there are no gifts exchanged for $X^l_{FF}$, so $g_1 = g_2 = 0$ on the LHS. On the RHS, gifts ARE exchanged because player 2 is a high type and the low type of player 1 lied. So we can set $g_1 = g_2 = g_M$:
			\begin{multline*}
				W\left(1 \right)-D\left(1 \right) \geq
			\left(1-\de^l\right) \left[T\left(1 + \al_2eg_M\right) + W\left(1 + \left(1-\al_1\right)eg_M\right) \right] \\
			+ \de^l \left(W\left(1 + \left(1-\al_1\right)eg_M\right)-D\left(1 + \left(1-\al_2\right)eg_M\right)\right) -pg_M+peg_M
			\end{multline*}
		\item With $a_1 = 0$ and $a_2 = 1$ (Recall the discussion above Equation~\ref{eqn:alphas}), we have
			\[
				W - D \geq \left(1-\de^l\right) \left[T\left(1 + eg_M\right) + W\left(1 + eg_M\right) \right] 
			+ \de^l \left(W\left(1 + eg_M\right) - D\right) -pg_M+peg_M
			\]
		\item It's time to start rearranging:
			\[
				pg_M\left(1 - e\right) \geq \left(1-\de^l\right) \left[T\left(1 + eg_M\right) + W\left(1 + eg_M\right) \right] 
			+ \de^l W\left(1 + eg_M\right) - \de^l D - W + D
			\]
			\[
				pg_M\left(1 - e\right) \geq \left(1-\de^l\right) \left[T\left(1 + eg_M\right) + W\left(1 + eg_M\right) \right] 
			+ \de^l W\left(1 + eg_M\right) + \left(1 - \de^l\right) D - W
			\]
			\begin{multline*}
			  	pg_M\left(1 - e\right) \geq T\left(1 + eg_M\right) + W\left(1 + eg_M\right) -\de^l T\left(1 + eg_M\right) -\de^l W\left(1 + eg_M\right) \\
			+ \de^l W\left(1 + eg_M\right) + \left(1 - \de^l\right) D - W
			\end{multline*}
			\[
				pg_M\left(1 - e\right) \geq T\left(1 + eg_M\right) + W\left(1 + eg_M\right) -\de^l T\left(1 + eg_M\right) + \left(1 - \de^l\right) D - W
			\]
			\[
				pg_M\left(1 - e\right) \geq T\left(1 + eg_M\right) + Weg_M -\de^l T\left(1 + eg_M\right) + \left(1 - \de^l\right) D
			\]
			\begin{equation}
				pg_M\left(1 - e\right) \geq \left(1 - \de^l\right) T\left(1 + eg_M\right) + Weg_M + \left(1 - \de^l\right) D
				\label{eq:examine}
			\end{equation}
	\end{itemize}

\subsubsection{Is it possible to pick e and g to satisfy low-type IC?}

Here I explore Expression~\ref{eq:examine} to see when it's possible to pick a gift high enough and an $e$ low enough to satisfy the low type's incentive constraint. Continuing on from Expression~\ref{eq:examine}:
\[
	pg_M\left(1 - e\right) \geq \left(T - \de_l T \right) \left(1 + eg_M\right) + Weg_M + \left(1 - \de_l \right) D
\]
	\begin{itemize}
		\item Distributing AGAIN
			\[
				pg	\left(1 - e\right) \geq T + Teg - \de_l T - \de_l T eg + Weg + \left(1 - \de_l \right) D
			\]
			\[
				pg	\left(1 - e\right) \geq \left(1 - \de_l \right) T + \left(1 - \de_l \right) D + Teg - \de_l T eg + Weg
			\]
			\begin{equation}
				pg_M - pg_Me \geq \left(1 - \de_l \right) \left( T + D \right) + eg_M \left( T - \de_l T + W \right)
				\label{eq:split}
			\end{equation}
		\item From here we can solve either for the constraint in terms of $g$ or in terms of $e$
	\end{itemize}		
			
\subsubsection{Solve for g}			
			First solve for $g$ starting from Expression~\ref{eq:split}, simplifying so that $g_M = g$ and $\de_l = \de$:
					$$pg - pge \geq \left(1 - \de \right) \left( T + D \right) + eg \left( T - \de T + W \right)$$
					$$pg - pge - eg \left( T - \de T + W \right)\geq \left(1 - \de \right) \left( T + D \right)$$
					\begin{equation}
						g \left(p - e \left( p + T - \de T + W \right) \right) \geq \left(1 - \de \right) \left( T + D \right)
						\label{eq:g1}
					\end{equation}
					\begin{equation}
						g_M \geq \frac{\left(1 - \de_l \right) \left( T + D \right)}{p - e \left( p + T - \de_l T + W \right)}
						\label{eq:g2}
					\end{equation}
						\begin{itemize}
							\item $T+D$, $1 - \de $ non-negative by assumption $\Rightarrow$ RHS of Expression~\ref{eq:g1} non-negative
								\begin{itemize}
									\item If $\left(p - pe - e \left( T - \de T + W \right) \right)$ is negative, the only way the IC constraint in Expression~\ref{eq:g1} can hold is if $g<0$.
									\item SO WE MUST ASSUME $p - e \left( p + T - \de T + W \right)$ IS POSITIVE
								\end{itemize}
							\item In the complicated environment of the mechanism, I haven't actually shown that the minimum low type IC gift is optimal
							\item To compare to the optimal gift size in the other sections:
								\begin{itemize}
									\item In Section 3 (Concessions Separating Equilibrium): $p\left(T+D\right)$
									\item Material value with $e=0$ same as in Section 3: $p\left(T+D\right)$
									\item Material value with $e=1$:
										\begin{equation}
											g^1 \geq \frac{p(1-\de_l)(T + D)}{\left(1-\de_l \right) -p(1-\de_l)T +(1-p)D}
											\label{eq:g4}
										\end{equation}
									\item Under the assumption that payoffs are multiplied by something linear like $1 + \al_2 g_1$, there's no reason to choose any intermediate value of $e$. The mediator, of course, can specify an intermediate value.
									\item We can simplify to see that the minimum gift when $e=1$ is larger when
										\[
											(1-\de)pT > (1-p)D
										\]
								\end{itemize}
						\end{itemize}
				
\subsubsection{Solve for e}
				Now solve for $e$ starting from Expression~\ref{eq:split}:
					$$pg - \left(1 - \de \right) \left( T + D \right) \geq peg + eg \left( T - \de T + W \right)$$
					\begin{equation}
						pg - \left(1 - \de \right) \left( T + D \right) \geq eg \left( p + T - \de T + W \right)
						\label{eq:g5}
					\end{equation}
						\begin{itemize}
							\item $p, 1-\de, T$ and $W$ are all assumed non-negative and $T+D > W$ so that $T$ is strictly positive. Since we also assume that gifts are always non-negative, $g \left( p + T - \de T + W \right)$ is non-negative
								\begin{itemize}
									\item For the IC constraint to hold then (with non-negative gifts), we must at least have $pg - \left(1 - \de \right) \left( T + D \right)$ non-negative.
								\end{itemize}
							\item If $g \left( p + T - \de T + W \right)$ is non-zero, we can divide through to get
								$$\frac{pg - \left(1 - \de \right) \left( T + D \right)}{g \left( p + T - \de T + W \right)} \geq e$$
						\end{itemize}
						
I did a lot of work in Mathematica to try to find out when the welfare under the mediator is better than the welfare under either the $e=0$ or $e=1$ concessions separating equilibria (CSE) [see Sections~\ref{sec:WM0} and \ref{sec:WM1}]
\begin{itemize}
	\item In the midst of this, I realized that, at the optimal $e$ (probably the largest one that satisfies the equation immediately above), both the gift and welfare $\rightarrow \infty$
		\begin{itemize}
			\item So we don't get a finite size of the gift or welfare value at the optimum with the low type IC constraint alone
				\begin{itemize}
					\item Interestingly, $g\cdot e$ (the effective gift) appears to be a line for a given set of parameters $W,T,D,\de_h,\de_l,p$
				\end{itemize}
		\end{itemize}
\end{itemize}

\subsubsection{High type incentive constraint}
\begin{equation}
	pX_{TT}^h+(1-p)X_{FF}^h - pg_h +peg_h \geq X_{FF}^h 
	%\label{eq:med4}
\end{equation}
Turns out to be same as high type individual rationality constraint.\\

If a high type lies, there would be at least one low type, and thus no gifts and they would fight forever. So there's no investment choice for high type who lies.
\begin{itemize}
	\item How to choose the optimal $e$ when both the low type and high type IC bind is still done informally in R.
\end{itemize}


								
\subsubsection{Welfare under Mediator vs. CSE with inefficient gifts}
\label{sec:WM0}
I tried to derive a necessary condition relating the IC/IR constraints to this welfare comparison but didn't find anything easily. I did this informally in Mathematica, but I don't think I kept the notes.

Here I'll try to derive a general result for when the welfare under a mediator would be higher than under $e=0$ separating with concessions equilibrium. \\

When $0<e<1$ under a mediator, the optimal $g_M$ is the one that chooses the highest welfare from a menu of $(e,g)$ pairs:
$$\frac{1}{1-\de_h}\left\{pT(1+eg_M) + (1-p)(W-D)\right\} - \frac{\left(1 - \de_l \right) \left( T + D \right)}{p - e \left( p + T - \de_l T + W \right)}$$

When $e=0$, welfare is $$\frac{1}{1-\de_h}\left\{pT + (1-p)(W-D)\right\} - p(T+D)$$
\textbf{THE ABOVE EQUATION IS wrong. We should subtract off $(1-p)g^0$ since $p$ of the time you receive a gift} \\

Ultimately I gave up on this because Mathematica couldn't solve the expressions with welfare under the mediator -- I believe this is because of the asymptote.
\begin{itemize}
	\item Instead, I looked for a sufficient condition to make welfare increase to the left of the asymptote (i.e. when the gift is positive).
\end{itemize}
			
\subsubsection{Welfare under Mediator vs. e=1}
\label{sec:WM1}
Here I'll try to derive a general result for when the welfare under a mediator would be higher than under $e=1$ separating with concessions equilibrium. \\

When $0<e<1$ under a mediator, the optimal $g_M$ is the one that chooses the highest welfare from a menu of $(e,g)$ pairs:
$$\frac{1}{1-\de_h}\left\{pT(1+eg_M) + (1-p)(W-D)\right\} - \frac{\left(1 - \de_l \right) \left( T + D \right)}{p - e \left( p + T - \de_l T + W \right)}$$

When $e=1$, welfare is $\frac{1}{1-\de_h}\left\{pT(1+g^1) + (1-p)(W-D(1+g^1))\right\} - \frac{p(1-d_l)(D+T)}{(1-d_l)(1-pT)+(1-p)D}$ \\
\textbf{THE ABOVE EQUATION IS wrong. We should subtract off $(1-p)g^1$ since $p$ of the time you receive a gift} \\

We compare them to find when the welfare under the mediator is larger:
	\begin{multline*}
		\frac{1}{1-\de_h}\left\{pT(1+eg_M) + (1-p)(W-D)\right\} - \frac{\left(1 - \de_l \right) \left( T + D \right)}{p - e \left( p + T - \de_l T + W \right)} \geq \\
		\frac{1}{1-\de_h}\left\{pT(1+g^1) + (1-p)(W-D(1+g^1))\right\} - \frac{p(1-d_l)(T+D)}{(1-d_l)(1-pT)+(1-p)D}
	\end{multline*}						
		
Eliminating common terms:
	\begin{multline*}
		\frac{1}{1-\de_h}\left\{pTeg_M \right\} - \frac{\left(1 - \de_l \right) \left( T + D \right)}{p - e \left( p + T - \de_l T + W \right)} \geq \\
		\frac{1}{1-\de_h}\left\{pTg^1 + (1-p)(- Dg^1))\right\} - \frac{p(1-d_l)(T+D)}{(1-d_l)(1-pT)+(1-p)D}
	\end{multline*}			
Rearranging:
	\begin{equation*}
		\frac{1}{1-\de_h}\left\{pTeg_M \right\} - g_M \geq \frac{1}{1-\de_h}\left\{pTg^1 + (1-p)(- Dg^1))\right\} - g^1
	\end{equation*}		
Rearranging again:
	\begin{equation*}
		pTeg_M - \left(1-\de_h\right)g_M \geq pTg^1 + (1-p)(- Dg^1)) - \left(1-\de_h\right)g^1
	\end{equation*}	
	\begin{equation*}
		g_M\left[pTe - \left(1-\de_h\right)\right] \geq pTg^1 -Dg^1 + pDg^1 - \left(1-\de_h\right)g^1
	\end{equation*}
	\begin{equation*}
		g_M\left[pTe - \left(1-\de_h\right)\right] \geq g^1\left[pT - D + pD - \left(1-\de_h\right)\right]
	\end{equation*}	
Substituting in the minimum gifts from Expression~\ref{eq:g2} and \ref{eq:g4}:				
	\begin{equation*}
		\frac{\left(1 - \de_l \right) \left( T + D \right)}{p - e \left( p + T - \de_l T + W \right)} \left[pTe - \left(1-\de_h\right)\right] \geq \frac{p(1-\de_l)(T + D)}{\left(1-\de_l \right)(1-pT) +(1-p)D} \left[pT - D + pD - \left(1-\de_h\right)\right]
	\end{equation*}	
Simplifying again because of common term in the numerators:
	\begin{equation*}
		\frac{1}{p - e \left( p + T - \de_l T + W \right)} \left[pTe - \left(1-\de_h\right)\right] \geq \frac{p}{\left(1-\de_l \right)(1-pT) +(1-p)D} \left[pT - D + pD - \left(1-\de_h\right)\right]
	\end{equation*}	
		\begin{equation*}
		\left[\left(1-\de_l \right)(1-pT) +(1-p)D\right] \left[pTe - \left(1-\de_h\right)\right] \geq \left[p - e \left( p + T - \de_l T + W \right)\right]p \left[pT - D + pD - \left(1-\de_h\right)\right]
	\end{equation*}	
	
I've used Mathematica to find a parameterization under which this is true.
\begin{itemize}
	\item I implemented the constraint in Mathematica to make sure the gift under $e=1$ is positive, BUT I did not implement the constraint on $\de_h$ that makes it IC (although I have a check in ``e-g calculations.R''. Here is the constraint on $\de_h$:
		\[
		  \de_h \geq D + 1 - p(T+D) + \frac{p(W-(T+D))}{g^1}
		\]
It turns out that there aren't really separate conditions for when the mediator is better in welfare terms than CSE with either efficient or inefficient gifts. If the welfare under the mediator asymptotes to positive infinity to the left of the asymptote (i.e. where the gift is positive), then the mediator is better than any CSE.
					\item I've given up on deriving an analytical expression for when the mediator enhances welfare (see Mathematica programs for more detail)
					\item Characterizing exactly the region where Welfare goes to positive $\infty$ to the left of the $e$ asymptote seems hard. But I have developed an intuitive sufficient condition
								\begin{itemize}
									\item First, note that it is not possible to have the $e$ asymptote be negative (in which case mediator can't ever work). The asymptote is at $e=\frac{p}{p + T(1-\de_l) +W}$. Everything is positive, and numerator strictly smaller than denominator, so $0 < e < 1$
									\item Both examining the IC constraint directly and taking a derivative in Mathematica give the following condition for when welfare under the mediator increases as $e$ increases; this is the case when the mediator is better than CSE $e=0$ or CSE $e=1$:
										\[
											\frac{e p T}{1 - \de_h} > 1
										\]
										\begin{itemize}
											\item This is a sufficient condition, not necessary. It only looks at the direct impact of the gift; there is another positive term that this ignores
												\begin{itemize}
													\item This makes welfare increasing in the gift, holding everything else constant, WHEN GIFT IS SET TO BE MINIMUM LOW-TYPE GIFT. I haven't show that's the optimal thing to do.
												\end{itemize}
											\item The welfare impact of the future value $(epT)$, summed over the whole future $\left(\frac{e p T}{1-h}g^M\right)$ has to be greater than the cost of giving the gift ($g^M$). 
										\end{itemize}
								\end{itemize}
\end{itemize}

\subsubsection{Check with some parameterizations}								
Parameterization that was in the previous version of the paper:
									\begin{itemize}
										\item Let $T = 10$, $W=8$ and $D = 5$
										\item Let $\de_h = .6$ and $\de_l = .3$. Assume $p = .2$. 
										\item Threshold for high types to separate when concessions do not have future material value is $\de = \frac{8}{10+5} = .5\overline{3}$. By Theorem 2, separation through concessions is possible.
										\item Minimum separating concession is $g^1 = \frac{p(1-d_l)(D+T)}{(1-d_l)(1-pT)+(1-p)D} = .\overline{63}$
										\item Minimum patience level required for separation is $\de_h^1 = D(1-p)+1-pT + \frac{p(W-(T+D))}{g^1} = .7\overline{9}.$ Because the specified $\de_h$ is not involved in any of these calculations, we can see that for any $\de_h \in [.5\overline{3},.7\overline{9})$, separating through concessions is possible when concessions have no material value but is not possible when they do have future material value.
										\item Maximum $e$ from Expression~\ref{eq:g5}:
											\[
											  0.2 - \frac{\left(1 - 0.3 \right) \left( 10 + 5 \right)}{g} \geq e \left( 0.2 + 10 - 0.3\cdot 10 + 8 \right)
											\]
											\[
											  0.2 - \frac{0.7 \left( 15 \right)}{g} \geq e \left( 11.2 \right)
											\]
											\[
											  \frac{0.2}{11.2} - \frac{0.7 \left( 15 \right)}{11.2g} \geq e 
											\]
											\[
											  0.01786 - \frac{9.375}{g} \geq e 
											\]
										To keep $e \geq 0$, we need $g \geq 525$ (this is REALLY big compared to $W$ for instance). It's also huge compared to the optimal gifts when $e=0$ of 3 and when $e=1$ of $0.\overline{63}$.
									\end{itemize}
								
\vskip.2in	
In the version sent to the JCR, I used three parameterizations; the calculations are done in ``e-g calculations.R'' Here, for instance, is Parameterization 2: $We > W1 > W0$, $54 > 7.33 > 7$. Mediator optimally chooses $e=0.24$ (0.25 is the asymptote). Optimal gift is 50, effective gift is 12, vs. $g1 = .667$ and $g0 = 1$
\begin{itemize}
	\item $T = 1, \ W = 1, \ D = 1, \ p = .5, \ \de_l = .5, \ \de_h = 15/16=0.9375$
\end{itemize}
	
						
						\vskip.3in
Welfare comparisons (always for high type)
							\begin{itemize}
								\item When $e=1$ (``W1''): $\frac{1}{1-\de_h}\left\{pT(1+g^*) + (1-p)(W-D(1+g^*))\right\} - (1-p)\frac{p(1-d_l)(D+T)}{(1-d_l)(1-pT)+(1-p)D}$
								\item When $e=0$ (``W0''): $\frac{1}{1-\de_h}\left\{pT + (1-p)(W-D)\right\} - (1-p)(T+D)$
								\item When $0<e<1$ under a mediator (``WM''): $\frac{1}{1-\de_h}\left\{pT(1+eg_M) + (1-p)(W-D)\right\} - (1-e)g_M$
									\begin{itemize}
										\item Optimal $g_M$: Expression~\ref{eq:g2} creates a menu of $(e,g)$ pairs that satisfy the constraint:
											\[
												g_M \geq \frac{\left(1 - \de \right) \left( T + D \right)}{p - e \left( p + T - \de T + W \right)}
											\]
										\item We're assuming $p - e \left( p + T - \de T + W \right) > 0$ (assuming gifts must be non-negative)
										\item Also $pg - \left(1 - \de \right) \left( T + D \right) \geq 0$ ($e$ version; this is WEAKER than the condition on $e$)
										
									\end{itemize}
								\item It's going to be very hard to find WM as intermediate. It has to be a case where it asymptotes to $-\infty$ and choosing $e=0$ is optimal anyway.
							\end{itemize}

							
\subsubsection{Notes}		

\begin{itemize}
	\item I'm ignoring the following question for the JCR submission: When $g^1$ turns negative, is that exactly when $e=1$ can't be sustained? It's not about $g^1$, but the $\de_h^1$ condition.
		\begin{itemize}
			\item When $g^1$ is negative, the constraint on $\de_h^1$ (see proof of Thm 3) becomes an upper bound. If it's negative, it means that no $\de_h$ exists to make separating possible.
			\item I need a formal result to see clearly what is going on here
		\end{itemize}
	\item I'm leaving the following out since we're already over the word limit and it's the least interesting of the results
		\begin{itemize}			
			\item Mediator with $e=0$ can be better than CSE with $e=0$. Need to say why this is
						\begin{itemize}
							\item Two differences: Mediator gets you some value of the gift $pTeg_M$; gift under mediator typically has to be larger--it takes more to disincentivize low type from lying because the payoff is better
						\end{itemize}
		\end{itemize}
\end{itemize}
			





\section{Would negotiating parties choose intermediate e in CSD?}

The question is this: if $\de^c < \de_h < \de^1$, would the high types ever want to choose an intermediate value of $e$? The idea is that $0<e<1$ could loosen the incentive constraint from the $e=0$ case.\\

High type's IC with potentially interior $e$ \\
\textbf{AGAIN, I don't have the gift right: you also receive the gift $p$ of the time; so everything below is off}
$$pT(1+eg) + (1-p)(W-D(1+eg)) - (1-\de_h)g \geq p(W+Weg-D) + (1-p)(W-D)$$
$$pT(1+eg) - (1-p)Deg - (1-\de_h)g \geq p(W+Weg-D)$$
$$\de_h g \geq p(W+Weg-D) - pT(1+eg) + (1-p)Deg + g$$
$$\de_h \geq \frac{p(W-D-T) + peg (W-T) + (1-p)Deg + g }{g}$$
$$\de_h \geq \frac{p(W-D-T)}{g} + p e(W-T) + (1-p)eD + 1$$

Now, what would $g$ need to be?
Substituting into Expression eq:matval1 from draft.tex and simplifying, we have
\begin{multline*}
	peg + \frac{1}{1-\de_l}\left[p(W+Weg-D) + (1-p)(W-D) \right] \geq \\
-g + peg + p\left(T + Teg +W + Weg + \frac{\de_l}{1-\de_l}\left(W + Weg-D \right) \right) + (1-p)\left( \frac{1}{1-\de_l}\left(W-D-Deg \right) \right)
\end{multline*}
\begin{equation*}
	-\frac{pD}{1-\de_l} \geq -g + p\left(T + Teg +\frac{\de_l}{1-\de_l}\left(-D \right) \right) + \left( \frac{1-p}{1-\de_l}\left(-Deg \right) \right)
\end{equation*}
\begin{equation*}
	(1-\de_l) g-pD \geq p\left((1-\de_l)(T + Teg) -\de_l D \right) + \left( (1-p)\left(-Deg \right) \right)
\end{equation*}
\begin{equation*}
	(1-\de_l) g\geq p(1-\de_l)(T + Teg+D) + \left( (1-p)\left(-Deg \right) \right)
\end{equation*}
\begin{equation*}
	(1-\de_l) g + (1-p)\left(Deg \right)-p(1-\de_l)Teg\geq p(1-\de_l)(T +D) 
\end{equation*}
\begin{equation*}
	g \left[(1-\de_l)(1-pTe) + (1-p)De \right]\geq p(1-\de_l)(T +D) 
\end{equation*}
As long as $(1-\de_l)(1-pTe) + (1-p)De>0$, we have
\begin{equation*}
	g \geq \frac{p(1-\de_l)(T +D) }{(1-\de_l)(1-pTe) + (1-p)De}
\end{equation*}
Compared to the constraint on the gift when $e=1$, when $e<1$, $-pTe$ makes the denominator larger and $De$ makes the denominator smaller. Not clear what the balance will be.
\begin{itemize}
	\item If the constraint gets tighter, you have to increase $g$.
	\item But remember that if the constraint gets looser, you can choose same $g$ if you want.
\end{itemize}
It's therefore not clear that we should just stick $g = \frac{p(1-\de_l)(T +D) }{(1-\de_l)(1-pTe) + (1-p)De}$ into the incentive constraint
\begin{itemize}
	\item For a particular example, we can wade through this to find the optimal $e$, but it might also involve letting $g$ get larger
	\item This is something I never considered in the $e=1$ and $e=0$ cases. I don't think it matters in $e=0$ case (check), but it might in $e=1$ case 
\end{itemize}

\vskip.5in
\section{Develop result: mediator increases potential for peace}
\un{June 6, 2017; deprecated since this mechanism isn't IC for the low type}

\begin{itemize}
	\item i.e. want to show mediator makes (Trust, Trust) an equilibrium over a larger parameter space than without mediator
\end{itemize}

\vskip.2in
What do we know?
\begin{itemize}
	\item Theorem 4: Under some parameters, optimal concessions aren't made when there isn't trust
		\begin{itemize}
			\item This is still a separating equilibrium, but completely inefficient concessions are given. So welfare is lower because of inefficient concessions. 
			\item Where there is a possibility of mediation helping to achieve peace where it otherwise would not be attainable (not just improving welfare) is if this reduction in welfare due to the inefficient concessions means that for some parameters it's not worth separating so we don't get peace at all without the mediator.
				\begin{itemize}
					\item i.e. when welfare for high type from separating falls below that for pooling, separating is no longer an equilibrium (convo with Jean-Guillaume).
				\end{itemize}
		\end{itemize}
	\item Theorem 5: Mediator eliminates inefficient concessions
		\begin{itemize}
			\item Here, we're already in that parameter space where there \emph{are} inefficient concessions
		\end{itemize}
\end{itemize}

\vskip.4in
Need to start by assuming that $\de_h <$ threshold where separating with no concessions works.
\begin{itemize}
	\item \textbf{Check to see whether Theorem 4 proof has to change.}
\end{itemize}

\vskip.2in
When $e=1$ and no material value, $g_h^* = p(T+D)$ is the smallest concession for CSE. \\
When $e=1$ and concessions have material value, $g' = \frac{p(1-\de_l)(D+T)}{(1-\de_l)(1-pT) + (1-p)D}$ is the minimum separating concession [Theorem 3]
I claim in Theorem 5 that the minimum separating concession under mediation is $g^* = \frac{g'}{1-p}$.

\vskip.2in
With $\al = 1$,
\begin{itemize}
	\item SWOC (separating without concessions) when $\de_h \geq \frac{W}{(1-p)W + p(T+D)}$
	\item CSE (concessions separating equilibrium) when $\de_h \geq \frac{W}{T+D}$
\end{itemize}
With $\al \in [0,1)$ and $e=1$,
\begin{itemize}
	\item SWOC doesn't change because there are no concessions
	\item WTS there are parameters where can't get STC outcome because of $\al <1$
		\begin{itemize}
			\item $e < 1$ means some of these \textit{can} get to peace
			\item but some can't under \textit{any} value of $e$ [assuming $e$ chosen at same time as $g$]
		\end{itemize}
\end{itemize}
Remember that decision between equilibria is made not by what \textit{can} be achieved in terms of $\de$, but by best response, i.e. ``Does high type do better by not giving concession?''
\begin{itemize}
	\item Should check in my numerical examples
\end{itemize}

\vskip.4in
Intermediate step for Theorem 4
\begin{itemize}
	\item Low type IC constraint to solve for equilibrium high type gift
	\item Already having shown $g_l = 0$, can set $g_h = g$ for simplicity of notation
	\item Also, since all the discount factors are for the low type, I'll let $\de_l = \de$
	\item For the case where $e=1$:

\end{itemize}
IC constraint:
\[
  X_{FF}^l \geq p X_{FT}^l + (1-p) X_{FF}^l - g
\]
Expanded in basic terms (without efficiency issues)
\[
  \frac{W-D}{1-\de} \geq p \left[T + W + \frac{\de}{1-\de}\left(W-D\right)\right] + (1-p) \frac{W-D}{1-\de} - g
\]
Multiply through by $\left(1-\de \right)$
\[
  W-D \geq p\left(1-\de \right) \left[T + W\right] + p\de \left(W-D\right) + (1-p) \left[W-D\right] - \left(1-\de \right) g
\]
Now add complexity from Table 2
\begin{multline*}
  W(1+(1-\al_1)g_2)-D(1+(1-\al_2)g_1) \geq \\
	p\left(1-\de \right) \left[T(1+\al_2 g_1) + W(1+(1-\al_1)g_2)\right] + p\de \left(W(1+(1-\al_1)g_2)-D(1+(1-\al_2)g_1)\right) \\ + (1-p) \left[W(1+(1-\al_1)g_2)-D(1+(1-\al_2)g_1)\right] - \left(1-\de \right) g
\end{multline*}
Substitute in $\al_1 = 0$ everywhere since this is the IC for a low-type of player 1:
\begin{multline*}
  W(1+g_2) - D(1+(1-\al_2)g_1) \geq \\
	p\left(1-\de \right) \left[T(1+\al_2 g_1) + W(1+g_2) \right] + p\de \left(W(1+g_2) - D(1+(1-\al_2)g_1)\right) \\ + (1-p) \left[W(1+g_2) - D(1+(1-\al_2)g_1)\right] - \left(1-\de \right) g
\end{multline*}
Set $g_1 = 0$ on the LHS and $g_1 = g$ on the RHS:
\begin{multline*}
  W(1+g_2) - D \geq \\
	p\left(1-\de \right) \left[T(1+\al_2 g) + W(1+g_2) \right] + p\de \left(W(1+g_2) - D(1+(1-\al_2)g)\right) \\ + (1-p) \left[W(1+g_2) - D(1+(1-\al_2)g)\right] - \left(1-\de \right) g
\end{multline*}
Set $g_2 = 0$ and $\al_2 = 0$ wherever there is a $(1-p)$ and $g_2 = g$ and $\al_2 = 1$ wherever there is a $p$:
\begin{multline*}
  pW(1+g) + (1-p)W - D \geq \\
	p\left(1-\de \right) \left[T(1+g) + W(1+g) \right] + p\de \left(W(1+g) - D\right) \\ + (1-p) \left[W - D(1+g)\right] - \left(1-\de \right) g
\end{multline*}
Expand
\begin{multline*}
  pW + pWg + W - pW - D \geq \\
	\left(p-p\de \right) \left[T+Tg + W+Wg \right] + p\de \left(W+Wg - D\right) \\ + (1-p) \left[W - D -Dg\right] - \left(1-\de \right) g
\end{multline*}
Cancel some like terms and move $\left(1-\de \right) g$ to LHS
\begin{multline*}
  \left(1-\de \right) g + pWg + W - D \geq \\
	p\left[T+Tg + W+Wg \right] -p\de \left[T+Tg \right] - p\de D \\ + (1-p) \left[W - D -Dg\right] 
\end{multline*}
Do some more canceling and expanding
\begin{multline*}
  \left(1-\de \right) g + W - D \geq \\
	p\left[T+Tg + W\right] -p\de T -p\de Tg - p\de D \\ + W - D -Dg -pW +pD +pDg
\end{multline*}
\[
  \left(1-\de \right) g \geq 
	pT+pTg + pW -p\de T -p\de Tg - p\de D  -Dg -pW +pD +pDg
\]
\[
  \left(1-\de \right) g \geq pT+pTg -p\de T -p\de Tg - p\de D  -Dg +pD +pDg
\]
Now just rearrange to get all the $g$ terms on the left
\[
  \left(1-\de \right) g -pTg + p\de Tg + Dg - pDg\geq pT -p\de T - p\de D +pD
\]
\[
  \left[\left(1-\de \right) -p(1-\de)T +(1-p)D\right]g \geq p(1-\de)(T + D)
\]
\[
  g \geq \frac{p(1-\de)(T + D)}{\left(1-\de \right) -p(1-\de)T +(1-p)D}
\]
In comparing to the minimum separating gift when $e=0$, which is
\[
  g^* \geq \frac{p(1-\de)(T + D)}{\left(1-\de \right)} = p(T + D)
\]
We can simplify to see that the minimum gift when $e=1$ is larger when
\[
  (1-\de)pT > (1-p)D
\]

\vskip.5in
Next we get the threshold for $\de_h$ that is necessary for a concessions separating eqm to exist:
\begin{itemize}
	\item Use the high-type IC constraint:
		\[
		  p X^h_{TT} + (1-p) X^h_{FF} - g_h \geq X^h_{FF}
		\]
	\item We'll need to expand for material effects but then solve for both $e=0$ and $e=1$
	\item For $e=1$:
		\[
		  \frac{p}{1-\de_h}T(1+g) + \frac{1-p}{1-\de_h}(W - D(1+g)) - g_h \geq \frac{1}{1-\de_h}(W-D)
		\]
		\[
		  pT + pTg + (1-p)(W - D - Dg) - g_h(1-\de_h) \geq W-D
		\]
		\[
		  pT + pTg + W - D - Dg - pW + pD + pDg - g_h + g_h \de_h \geq W-D
		\]
		\[
		  pT + pTg - Dg - pW + pD + pDg - g_h + g_h \de_h \geq 0
		\]
		\[
		  pT + pTg - Dg - pW + pD + pDg - g_h + g_h \de_h \geq 0
		\]
		Note that all the $g$'s should be $g_h$'s in the five lines above:
		\[
		  g_h \de_h \geq - pT - pT g_h + D g_h + pW - pD - pD g_h + g_h
		\]
		\[
		  \de_h \geq \frac{p(W - D - T)}{g_h} + (1-p)D + 1 - pT 
		\]
	\item For $e=0$, you're back in the original case with no material value, i.e. enforceable for same $\de_h$ as no material value case
		\[
		  \frac{p}{1-\de_h}T + \frac{1-p}{1-\de_h}(W - D) - g_0 \geq \frac{1}{1-\de_h}(W-D)
		\]
		\[
		  pT + (1-p)(W - D) - (1-\de_h)g_0 \geq (W-D)
		\]
		\[
		  pT -p(W - D) + \de_h g_0 \geq g_0
		\]
		\[
		  \de_h g_0 \geq g_0 + p(W - D - T) 
		\]
		\[
		  \de_h \geq 1 + \frac{p(W - D - T)}{g_0}
		\]
		Optimal separating gift in this setting is $g_0 = p(D+T)$. Substituting this in, we have:
		\[
		  \de_h \geq 1 + \frac{p(W - D - T)}{p(D+T)} = 1 + \frac{pW - p(D - T)}{p(D+T)} = 1+\frac{pW}{p(D+T)}-1 = \frac{W}{D+T}
		\]
		
		\begin{itemize}
			\item But check: when $e=0$, also wipe out immediate monetary value? YES, in terms of benefit, but NOT cost, which is the only part that doesn't cancel out of IC constraints (same for high type and low type; benefit is on both sides; cost is on only one side)
				\begin{itemize}
					\item Not from utility standpoint, but from IC standpoint because direct benefit of gift cancels out
				\end{itemize}
		\end{itemize}
	\item But high type utility goes down
		\begin{itemize}
			\item Interestingly, if $e=0$, if high types would have separated without material value, they will here with material value that is destroyed (they don't do better in pooling). We see this by calculating the high type utility for the case where $e=0$, which is
				\[
				  \frac{1}{1-\de_h}\left[pT + (1-p)(W-D) \right] - g_0
				\]
				Substituting in the value of the $g_0$ (the optimal separating gift with $e=0$, which I have as $p(T+D)$ but I should check again to make sure it's correct for this situation) and the threshold value to separate when $e=0$ (i.e. $\de = \frac{W}{T+D}$), we have
				\[
				  \frac{T+D}{T+D-W}\left[pT + (1-p)(W-D) \right] - p(T+D)
				\]
				\[
				  \frac{T+D}{T+D-W}\left[p(T +D -W) + (W-D) \right] - p(T+D)
				\]
				\[
				  p(T+D) + \frac{T+D}{T+D-W}\left[W-D \right] - p(T+D)
				\]
				\[
				  \frac{T+D}{T+D-W}\left[W-D \right]
				\]
				This is exactly the payoff to the pooling equilibrium (that is, both types pool on fight with no gift giving) when $\de = \frac{W}{T+D}$, since $U_{pooling} = \frac{W-D}{1-\de}$.
				
				So $U_{g_0} = U_{pooling}$ at the cutoff for separating, and $U_{g_0}$ is larger for any discount factor higher than the cutoff. 
					\begin{itemize}
						\item This can be seen, for instance, by decomposing $U_{pooling}$ along the same lines as $U_{g_0}$ above into components for the cooperation vs. punishment and cost of the gift (even though that's not what's going on in pooling). The whole thing increases, including the cost of the gift portion for the pooling when $\de$ increases. But the cost of the gift does not scale up in $U_{g_0}$.
					\end{itemize}
		\end{itemize}
		We can also do this more generally. Take
			\[
			  U_{g_0} = \frac{1}{1-\de}\left(W-D\right)
			\]
			\[
				U_{pooling} = \frac{1}{1-\de_h}\left[pT + (1-p)(W-D) \right] - p(T+D) 
			\]
			\[
			  = pT + pD +\frac{\de}{1-\de}\left(pT + pD\right) + \frac{1}{1-\de} \left( W -D -pW\right) - pT -pD
			\]
			\[
			  = \frac{\de}{1-\de}\left(pT + pD\right) + \frac{1}{1-\de} \left( W -pW -D\right)
			\]
		Now compare the two, $U_{g_0} \lessgtr U_{pooling}$:
				\[
				  \frac{\de}{1-\de}\left(pT + pD\right) + \frac{1}{1-\de} \left( W -pW -D\right) \lessgtr \frac{1}{1-\de}\left(W-D\right)
				\]
				\[
				  \frac{\de}{1-\de}\left(pT + pD\right) - \frac{1}{1-\de} \left( pW \right) \lessgtr 0
				\]
				\[
				  \de\left(pT + pD\right) \lessgtr pW 
				\]
				\[
				  \de \lessgtr \frac{pW}{pT + pD} = \frac{W}{T+D}
				\]
			So that the utility from $U_{g_0}$ is greater exactly when the discount factor is above the threshold for concessions separating equilibrium to be possible without material value.
\end{itemize}

\vskip.2in
How does utility compare under $e=1$ and $e=0$? Compare the two, $U_{g_1} \lessgtr U_{g_0}$:
				\[
				  \frac{p}{1-\de_h}T(1+g) + \frac{1-p}{1-\de_h}(W - D(1+g)) - g_h \lessgtr \frac{1}{1-\de_h}\left[pT + (1-p)(W-D) \right] - g_0
				\]
				\[
				  pT(1+g_1) + \left(1-p\right)(W - D(1+g_1)) - \left(1-\de_h\right)g_1 \lessgtr pT + (1-p)(W-D) - \left(1-\de_h\right)g_0
				\]
				\[
				  pTg_1 + \left(1-p\right)(W - D -Dg_1) - \left(1-\de_h\right)g_1 \lessgtr (1-p)(W-D) - \left(1-\de_h\right)g_0
				\]
				\[
				  pTg_1 - \left(1-p\right)Dg_1 - \left(1-\de_h\right)g_1 \lessgtr - \left(1-\de_h\right)g_0
				\]
				\[
				  pTg_1 - \left(1-p\right)Dg_1  \lessgtr \left(1-\de_h\right)g_1 - \left(1-\de_h\right)g_0
				\]
				\begin{equation}
				  pT - \left(1-p\right)D  \lessgtr \frac{\left(1-\de_h\right)\left(g_1 - g_0 \right)}{g_1}
					\label{eq:util}
				\end{equation}

We know (from above) that the minimum gift when $e=1$ is larger than the minimum gift when $e=0$ when
\[
  (1-\de)pT > (1-p)D
\]

\vskip.2in
Let's case this out.
\begin{enumerate}
	\item Assume $g_0 = g_1$; then $(1-\de)pT = (1-p)D$. The right hand side of \ref{eq:util} is 0 and the left hand side is
		\[
			pT - (1-\de)pT = pT - pT + \de pT = \de pT 
		\]
		And utility under $e=1$ is higher.
	\item When $g_0 > g_1$; then $(1-\de)pT < (1-p)D$. The right hand side of \ref{eq:util} is negative and the left hand side is larger (more positive) than in case 1, so utility under $e=1$ is also higher in this case.
	\item When $g_1 > g_0$; then $(1-\de)pT > (1-p)D$. The right hand side of \ref{eq:util} becomes positive and the left hand side becomes smaller (less positive, and possibly negative) than in case 1, so utility under $e=1$ is higher when $g_1$ is only a little larger than $g_0$, but eventually as $g_1$ gets much larger than $g_0$, utility under $e=0$ becomes larger than utility under $e=1$.
\end{enumerate}


\newpage
\section{To-do list after Venice CesIfo 06/15/17}
\begin{itemize}
	\item Jim Fearon: maybe get rid of repeated game and just parameterize a one-shot game (doens't work because concessions required for separation are too large; require future to recoup cost)
		\begin{itemize}
			\item he has a paper about concessions being used against you; he also gave me the reference for another one but I've forgotten it
		\end{itemize}
\end{itemize}

\section{Notes from CPEG, Oct. 25, 2019}
\begin{itemize}
	\item Arnaud
		\begin{itemize}
			\item $\de$ are not exogenous (look at Brexit)
			\item Proposes a one-shot game to replace the repeated game. Simple bayesian game with difference preferences depening on type.
		\end{itemize}
	\item Raphael Godefroy
		\begin{itemize}
			\item Can $\de_i$ be a function of $g$? That is, likelihood of defection decreases as concession gets better
		\end{itemize}
\end{itemize}


\end{document}