\documentclass[12pt]{article}

\addtolength{\textwidth}{1.4in}
\addtolength{\oddsidemargin}{-.7in} %left margin
\addtolength{\evensidemargin}{-.7in}
\setlength{\textheight}{8.5in}
\setlength{\topmargin}{0.0in}
\setlength{\headsep}{0.0in}
\setlength{\headheight}{0.0in}
\setlength{\footskip}{.5in}
\renewcommand{\baselinestretch}{1.0}
\setlength{\parindent}{0pt}
\linespread{1.1}

\usepackage[pdftex,
bookmarks=true,
bookmarksnumbered=false,
pdfview=fitH,
bookmarksopen=true]{hyperref}

\usepackage{amssymb, amsmath, amsthm, bm}
\usepackage{graphicx,csquotes,verbatim}
\usepackage[backend=biber,block=space,style=authoryear]{biblatex}
\setlength{\bibitemsep}{\baselineskip}
\usepackage[american]{babel}
%dell laptop
\addbibresource{C:/Users/Kristy/Dropbox/Research/xBibs/tradeagreements.bib}
%\addbibresource{C:/Users/Kristy/Documents/Dropbox/Research/xBibs/tradeagreements.bib}
\renewcommand{\newunitpunct}{,}
\renewbibmacro{in:}{}


\DeclareMathOperator*{\argmax}{arg\,max}
\usepackage{xcolor}
\hbadness=10000

\newtheorem{proposition}{Proposition}
\newcommand{\ve}{\varepsilon}
\newcommand{\ov}{\overline}
\newcommand{\un}{\underline}
\newcommand{\ta}{\theta}
\newcommand{\al}{\alpha}
\newcommand{\Ta}{\Theta}
\newcommand{\expect}{\mathbb{E}}
\newcommand{\Bt}{B(\bm{\tau^a})}
\newcommand{\bta}{\bm{\tau^a}}
\newcommand{\btn}{\bm{\tau^n}}
\newcommand{\btw}{\bm{\tau^{tw}}}
\newcommand{\ga}{\gamma}
\newcommand{\Ga}{\Gamma}
\newcommand{\de}{\delta}

\begin{document}
\begin{center}
  Working notes for Inefficient Concessions
\end{center}

\vskip.3in
\un{June 6, 2017}\\
Want to develop result showing that mediator increases potential for peace
\begin{itemize}
	\item i.e. want to show mediator makes (Trust, Trust) an equilibrium over a larger parameter space than without mediator
\end{itemize}

\vskip.2in
What do we know?
\begin{itemize}
	\item Theorem 4: Under some parameters, optimal concessions aren't made when there isn't trust
		\begin{itemize}
			\item This is still a separating equilibrium, but completely inefficient concessions are given. So welfare is lower because of inefficient concessions. 
			\item Where there is a possibility of mediation helping to achieve peace where it otherwise would not be attainable (not just improving welfare) is if this reduction in welfare due to the inefficient concessions means that for some parameters it's not worth separating so we don't get peace at all without the mediator.
				\begin{itemize}
					\item i.e. when welfare for high type from separating falls below that for pooling, separating is no longer an equilibrium (convo with Jean-Guillaume).
				\end{itemize}
		\end{itemize}
	\item Theorem 5: Mediator eliminates inefficient concessions
		\begin{itemize}
			\item Here, we're already in that parameter space where there \emph{are} inefficient concessions
		\end{itemize}
\end{itemize}

\vskip.4in
Need to start by assuming that $\de_h <$ threshold where separating with no concessions works.
\begin{itemize}
	\item \textbf{Check to see whether Theorem 4 proof has to change.}
\end{itemize}

\vskip.2in
When $e=1$ and no material value, $g_h^* = p(T+D)$ is the smallest concession for CSE. \\
When $e=1$ and concessions have material value, $g' = \frac{p(1-\de_l)(D+T)}{(1-\de_l)(1-pT) + (1-p)D}$ is the minimum separating concession [Theorem 3]
I claim in Theorem 5 that the minimum separating concession under mediation is $g^* = \frac{g'}{1-p}$.

\vskip.2in
With $\al = 1$,
\begin{itemize}
	\item SWOC (separating without concessions) when $\de_h \geq \frac{W}{(1-p)W + p(T+D)}$
	\item CSE (concessions separating equilibrium) when $\de_h \geq \frac{W}{T+D}$
\end{itemize}
With $\al \in [0,1)$ and $e=1$,
\begin{itemize}
	\item SWOC doesn't change because there are no concessions
	\item WTS there are parameters where can't get STC outcome because of $\al <1$
		\begin{itemize}
			\item $e < 1$ means some of these \textit{can} get to peace
			\item but some can't under \textit{any} value of $e$ [assuming $e$ chosen at same time as $g$]
		\end{itemize}
\end{itemize}
Remember that decision between equilibria is made not by what \textit{can} be achieved in terms of $\de$, but by best response, i.e. ``Does high type do better by not giving concession?''
\begin{itemize}
	\item Should check in my numerical examples
\end{itemize}

\vskip.4in
Intermediate step for Theorem 4
\begin{itemize}
	\item Low type IC constraint to solve for equilibrium high type gift
	\item Already having shown $g_l = 0$, can set $g_h = g$ for simplicity of notation
	\item Also, since all the discount factors are for the low type, I'll let $\de_l = \de$
	\item For the case where $e=1$:

\end{itemize}
IC constraint:
\[
  X_{FF}^l \geq p X_{FT}^l + (1-p) X_{FF}^l - g
\]
Expanded in basic terms (without efficiency issues)
\[
  \frac{W-D}{1-\de} \geq p \left[T + W + \frac{\de}{1-\de}\left(W-D\right)\right] + (1-p) \frac{W-D}{1-\de} - g
\]
Multiply through by $\left(1-\de \right)$
\[
  W-D \geq p\left(1-\de \right) \left[T + W\right] + p\de \left(W-D\right) + (1-p) \left[W-D\right] - \left(1-\de \right) g
\]
Now add complexity from Table 2
\begin{multline*}
  W(1+(1-\al_1)g_2)-D(1+(1-\al_2)g_1) \geq \\
	p\left(1-\de \right) \left[T(1+\al_2 g_1) + W(1+(1-\al_1)g_2)\right] + p\de \left(W(1+(1-\al_1)g_2)-D(1+(1-\al_2)g_1)\right) \\ + (1-p) \left[W(1+(1-\al_1)g_2)-D(1+(1-\al_2)g_1)\right] - \left(1-\de \right) g
\end{multline*}
Substitute in $\al_1 = 0$ everywhere since this is the IC for a low-type of player 1:
\begin{multline*}
  W(1+g_2) - D(1+(1-\al_2)g_1) \geq \\
	p\left(1-\de \right) \left[T(1+\al_2 g_1) + W(1+g_2) \right] + p\de \left(W(1+g_2) - D(1+(1-\al_2)g_1)\right) \\ + (1-p) \left[W(1+g_2) - D(1+(1-\al_2)g_1)\right] - \left(1-\de \right) g
\end{multline*}
Set $g_1 = 0$ on the LHS and $g_1 = g$ on the RHS:
\begin{multline*}
  W(1+g_2) - D \geq \\
	p\left(1-\de \right) \left[T(1+\al_2 g) + W(1+g_2) \right] + p\de \left(W(1+g_2) - D(1+(1-\al_2)g)\right) \\ + (1-p) \left[W(1+g_2) - D(1+(1-\al_2)g)\right] - \left(1-\de \right) g
\end{multline*}
Set $g_2 = 0$ and $\al_2 = 0$ wherever there is a $(1-p)$ and $g_2 = g$ and $\al_2 = 1$ wherever there is a $p$:
\begin{multline*}
  pW(1+g) + (1-p)W - D \geq \\
	p\left(1-\de \right) \left[T(1+g) + W(1+g) \right] + p\de \left(W(1+g) - D\right) \\ + (1-p) \left[W - D(1+g)\right] - \left(1-\de \right) g
\end{multline*}
Expand
\begin{multline*}
  pW + pWg + W - pW - D \geq \\
	\left(p-p\de \right) \left[T+Tg + W+Wg \right] + p\de \left(W+Wg - D\right) \\ + (1-p) \left[W - D -Dg\right] - \left(1-\de \right) g
\end{multline*}
Cancel some like terms and move $\left(1-\de \right) g$ to LHS
\begin{multline*}
  \left(1-\de \right) g + pWg + W - D \geq \\
	p\left[T+Tg + W+Wg \right] -p\de \left[T+Tg \right] - p\de D \\ + (1-p) \left[W - D -Dg\right] 
\end{multline*}
Do some more canceling and expanding
\begin{multline*}
  \left(1-\de \right) g + W - D \geq \\
	p\left[T+Tg + W\right] -p\de T -p\de Tg - p\de D \\ + W - D -Dg -pW +pD +pDg
\end{multline*}
\[
  \left(1-\de \right) g \geq 
	pT+pTg + pW -p\de T -p\de Tg - p\de D  -Dg -pW +pD +pDg
\]
\[
  \left(1-\de \right) g \geq pT+pTg -p\de T -p\de Tg - p\de D  -Dg +pD +pDg
\]
Now just rearrange to get all the $g$ terms on the left
\[
  \left(1-\de \right) g -pTg + p\de Tg + Dg - pDg\geq pT -p\de T - p\de D +pD
\]
\[
  \left[\left(1-\de \right) -p(1-\de)T +(1-p)D\right]g \geq p(1-\de)(T + D)
\]
\[
  g \geq \frac{p(1-\de)(T + D)}{\left(1-\de \right) -p(1-\de)T +(1-p)D}
\]
In comparing to the minimum separating gift when $e=0$, which is
\[
  g^* \geq \frac{p(1-\de)(T + D)}{\left(1-\de \right)}
\]
We can simplify to see that the minimum gift when $e=1$ is larger when
\[
  (1-\de)pT > (1-p)D
\]

\vskip.5in
Next we get the threshold for $\de_h$ that is necessary for a concessions separating eqm to exist:
\begin{itemize}
	\item Use the high-type IC constraint:
		\[
		  p X^h_{TT} + (1-p) X^h_{FF} - g_h \geq X^h_{FF}
		\]
	\item We'll need to expand for material effects but then solve for both $e=0$ and $e=1$
	\item For $e=1$:
		\[
		  \frac{p}{1-\de_h}T(1+g) + \frac{1-p}{1-\de_h}(W - D(1+g)) - g_h \geq \frac{1}{1-\de_h}(W-D)
		\]
		\[
		  pT + pTg + (1-p)(W - D - Dg) - g_h(1-\de_h) \geq W-D
		\]
		\[
		  pT + pTg + W - D - Dg - pW + pD + pDg - g_h + g_h \de_h \geq W-D
		\]
		\[
		  pT + pTg - Dg - pW + pD + pDg - g_h + g_h \de_h \geq 0
		\]
		\[
		  pT + pTg - Dg - pW + pD + pDg - g_h + g_h \de_h \geq 0
		\]
		Note that all the $g$'s should be $g_h$'s in the five lines above:
		\[
		  g_h \de_h \geq - pT - pT g_h + D g_h + pW - pD - pD g_h + g_h
		\]
		\[
		  \de_h \geq \frac{p(W - D - T)}{g_h} + (1-p)D + 1 - pT 
		\]
	\item For $e=0$, you're back in the original case with no material value, i.e. enforceable for same $\de_h$ as no material value case
		\[
		  \frac{p}{1-\de_h}T + \frac{1-p}{1-\de_h}(W - D) - g_0 \geq \frac{1}{1-\de_h}(W-D)
		\]
		\[
		  pT + (1-p)(W - D) - (1-\de_h)g_0 \geq (W-D)
		\]
		\[
		  pT -p(W - D) + \de_h g_0 \geq g_0
		\]
		\[
		  \de_h g_0 \geq g_0 + p(W - D - T) 
		\]
		\[
		  \de_h \geq 1 + \frac{p(W - D - T)}{g_0}
		\]
		Optimal separating gift in this setting is $g_0 = p(D+T)$. Substituting this in, we have:
		\[
		  \de_h \geq 1 + \frac{p(W - D - T)}{p(D+T)} = 1 + \frac{pW - p(D - T)}{p(D+T)} = 1+\frac{pW}{p(D+T)}-1 = \frac{W}{D+T}
		\]
		
		\begin{itemize}
			\item But check: when $e=0$, also wipe out immediate monetary value? YES, in terms of benefit, but NOT cost, which is the only part that doesn't cancel out of IC constraints (same for high type and low type; benefit is on both sides; cost is on only one side)
				\begin{itemize}
					\item Not from utility standpoint, but from IC standpoint because direct benefit of gift cancels out
				\end{itemize}
		\end{itemize}
	\item But high type utility goes down
		\begin{itemize}
			\item Interestingly, if $e=0$, if high types would have separated without material value, they will here with material value that is destroyed (they don't do better in pooling). We see this by calculating the high type utility for the case where $e=0$, which is
				\[
				  \frac{1}{1-\de_h}\left[pT + (1-p)(W-D) \right] - g_0
				\]
				Substituting in the value of the $g_0$ (the optimal separating gift with $e=0$) and the threshold value to separate when $e=0$, we have
				\[
				  \frac{T+D}{T+D-W}\left[pT + (1-p)(W-D) \right] - p(T+D)
				\]
				\[
				  \frac{T+D}{T+D-W}\left[p(T +D -W) + (W-D) \right] - p(T+D)
				\]
				\[
				  p(T+D) + \frac{T+D}{T+D-W}\left[W-D \right] - p(T+D)
				\]
				\[
				  \frac{T+D}{T+D-W}\left[W-D \right]
				\]
				This is exactly the payoff to the pooling equilibrium when $\de = \frac{W}{T+D}$. So $U_{g_0} = U_{pooling}$ at the cutoff for separating, and $U_{g_0}$ is larger for any discount factor higher than the cutoff. The value of the change of peace for ever is outweighed by having to pay for the gift even though the value of the gift will never be received.
		\end{itemize}
\end{itemize}

\newpage
To-do list after Venice CesIfo 06/15/17:
\begin{itemize}
	\item Jim Fearon: maybe get rid of repeated game and just parameterize a one-shot game (doens't work because concessions required for separation are too large; require future to recoup cost)
		\begin{itemize}
			\item he has a paper about concessions being used against you; he also gave me the reference for another one but I've forgotten it
		\end{itemize}

\vskip.3in
Notes from CPEG, Oct. 25, 2019 presentation:
\begin{itemize}
	\item Arnaud
		\begin{itemize}
			\item $\de$ are not exogenous (look at Brexit)
			\item Proposes a one-shot game to replace the repeated game. Simple bayesian game with difference preferences depening on type.
		\end{itemize}
	\item Raphael Godefroy
		\begin{itemize}
			\item Can $\de_i$ be a function of $g$? That is, likelihood of defection decreases as concession gets better
		\end{itemize}
\end{itemize}

\newpage
Review 1 from EJPE (Email from Toke Aidt, August 19, 2018): \\

While clearly these are interesting issues, I have difficulties reading this paper. Everything is presented in a very unclear manner. Game-theoretical concepts are invoked, but typically not correctly. For instance, an equilibrium is a strategy profile (one strategy for each type of each player). Hence it does not make sense to say, as the authors do in Theorem 1 on page 10, that something is an equilibrium strategy for one type without also specifying what others do. As another example, on page 18, the authors say that “The revelation principle is used to attain truthful self-identification.” But in reality, the revelation principle is just an argument for why it is without loss of generality to restrict attention to direct mechanisms. The truthfulness of messages is guaranteed by requiring the mechanism to be incentive compatible, something the authors do not address. \\

Another thing. It is not considered a good idea to introduce new assumptions, which have not been mentioned before, in the proof of a proposition, as happens in the alleged proof of Theorem 4 on page 27.

\end{itemize}
\end{document}