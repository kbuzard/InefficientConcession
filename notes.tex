\documentclass[12pt]{article}

\addtolength{\textwidth}{1.4in}
\addtolength{\oddsidemargin}{-.7in} %left margin
\addtolength{\evensidemargin}{-.7in}
\setlength{\textheight}{8.5in}
\setlength{\topmargin}{0.0in}
\setlength{\headsep}{0.0in}
\setlength{\headheight}{0.0in}
\setlength{\footskip}{.5in}
\renewcommand{\baselinestretch}{1.0}
\setlength{\parindent}{0pt}
\linespread{1.1}

\usepackage[pdftex,
bookmarks=true,
bookmarksnumbered=false,
pdfview=fitH,
bookmarksopen=true]{hyperref}

\usepackage{amssymb, amsmath, amsthm, bm}
\usepackage{graphicx,csquotes,verbatim}
\usepackage[backend=biber,block=space,style=authoryear]{biblatex}
\setlength{\bibitemsep}{\baselineskip}
\usepackage[american]{babel}
%dell laptop
\addbibresource{C:/Users/Kristy/Dropbox/Research/xBibs/tradeagreements.bib}
%\addbibresource{C:/Users/Kristy/Documents/Dropbox/Research/xBibs/tradeagreements.bib}
\renewcommand{\newunitpunct}{,}
\renewbibmacro{in:}{}


\DeclareMathOperator*{\argmax}{arg\,max}
\usepackage[dvipsnames]{xcolor}
\hbadness=10000

\newtheorem{proposition}{Proposition}
\newcommand{\ve}{\varepsilon}
\newcommand{\ov}{\overline}
\newcommand{\un}{\underline}
\newcommand{\ta}{\theta}
\newcommand{\al}{\alpha}
\newcommand{\Ta}{\Theta}
\newcommand{\expect}{\mathbb{E}}
\newcommand{\Bt}{B(\bm{\tau^a})}
\newcommand{\bta}{\bm{\tau^a}}
\newcommand{\btn}{\bm{\tau^n}}
\newcommand{\btw}{\bm{\tau^{tw}}}
\newcommand{\ga}{\gamma}
\newcommand{\Ga}{\Gamma}
\newcommand{\de}{\delta}

\begin{document}
\begin{center}
  Working notes for Inefficient Concessions
\end{center}

\vskip.3in

\section{Make mediator work}
November 29, 2020: In making revisions in April 2020 to respond to the EJPE referee reports, I discovered the IC constraint does not hold for the low type in Section 5: Mediation. Here I will see what adjustments might make the mediator work.
\begin{itemize}
	\item November 30: I showed formally and in great detail in Section \ref{sec:e1} that the IC constraint does not bind for $e=1$ with our standard assumptions.
	\item December 1: I began working on the $e < 1$ case in Section \ref{sec:eless}
\end{itemize}

\subsection{Mediation: Basic Details and Overview}
\begin{itemize}
	\item If the mediator gets two `high' reports, both countries are instructed to exchange gifts and play `Trust'
	\item If the mediator gets one or zero `high' reports, there is no gift and they both play `Fight'
\end{itemize}
I show the constraints for both types in Section \ref{sec:const}.\\
I detail the choices of all types in regards to investments in civil society in Section \ref{sec:civ}.\\
I show in Section \ref{sec:e1} that the low type's IC constraint cannot hold if $e=1$.\\


\subsection{IC constraints for Mediation}
\label{sec:const}
\begin{itemize}
	\item High type IC:
		$$pX_{TT}^h+(1-p)X_{FF}^h - pg_h +peg_h \geq X_{FF}^h $$
			\begin{itemize}
				\item Relative to Section 4.2, $-g_h$ there becomes $-pg_h$ here (again, only give gift if matched with high type). Also $+peg_h$ in Section 4.2 disappears on RHS.
			\end{itemize}
	\item Low type IC:
		$$X_{FF}^l \geq pX_{FT}^l+(1-p)X_{FF}^l-pg_h+peg_h $$
			\begin{itemize}
				\item If a low type reveals truthfully: you get $X_{FF}$ and no gift with probability $p$; you get the same with probability $(1-p)$ (you need both countries to be high types in order to do anything)
				\item If a low type lies: get $X_{FT}$ and give gift ($g_h$) and receive gift ($eg_h$) with probability $p$; $X_{FF}$ again with probability $(1-p)$
				\item Relative to Section 4.2, there is no $peg_h$ on the LHS (in Section 4.2, it cancels out with the $peg_h$ on the RHS): no benefit of getting the gift from the high type. Also, $-g_h$ on the RHS becomes $-pg_h$ since you only send the gift if matched with a high type
			\end{itemize}
\end{itemize}

\subsection{Civil society investment choices}
\label{sec:civ}
In a concessions separating equilibrium or in the mechanism, the high type always has the incentive to invest all of the gift in civil socity
	\begin{itemize}
		\item If there is an exchange of gifts (whether in the mechanism or a truthfully revealed separating equilibrium), you know you're both receiving from and giving to a high type
		\item \textbf{I should check this in all the scenarios to make sure we don't invoke it when it's not right}
			\begin{itemize}
				\item \textbf{I also want to know whether it depends on the linear form}
				\item \textbf{Make more explicit in the text}
			\end{itemize}
	\end{itemize}
$\al_i$ is the proportion of the gift that country $i$ receives that it invests in civil society.
	\begin{itemize}
		\item $\al_i = 1$ means the whole value of the gift is invested in civil society. High types make this choice
			\begin{itemize}
				\item $\al_i = 0$ means the whole value of the gift is invested in military buildup
			\end{itemize}
			
		\item High type of Country 1 chooses $\al_1=1$ (all into civil society) 
			\begin{itemize}
				\item $(1-\al_1) = 0$ (nothing into military)
			\end{itemize}
		\item Low type of Country 1 chooses $\al_1=0$ (nothing into civil society) 
			\begin{itemize}
				\item $(1-\al_1) = 1$ (all into military)
			\end{itemize}
		\item High type of Country 2 chooses $\al_2=1$ (all into civil society) 
			\begin{itemize}
				\item $(1-\al_2) = 0$ (nothing into military)
			\end{itemize}
		\item Low type of Country 2 chooses $\al_2=0$ (nothing into civil society) 
			\begin{itemize}
				\item $(1-\al_2) = 1$ (all into military)
			\end{itemize}	
	\end{itemize}



\subsection{Why does low type IC not hold with efficient gifts?}
\label{sec:e1}
\begin{equation}
	X_{FF}^l \geq pX_{FT}^l+(1-p)X_{FF}^l-pg_h+peg_h
	\label{eqn:IClow}
\end{equation}
With our restrictions (mostly the linear assumption, IIRC), High type gives gift with efficiency $e=1$, so gift giving and receiving cancels out. This leaves
		$$X_{FF}^l \geq pX_{FT}^l + X_{FF}^l - pX_{FF}^l$$
		which simplifies to 
		$$X_{FF}^l \geq X_{FT}^l.$$
		Expanding, we have
		\[
			\frac{W\left( \right)-D\left( \right)}{1-\de^l} \geq T\left( \right) + W\left( \right) + \frac{\de^l}{1-\de^l}\left(W\left( \right) -D\left( \right) \right)
		\]
		Now we have to fill in all the complicated bits in parentheses from Table 3 in Section 4.2
		\begin{multline}
			W\left(m_1 + \left(1-\al_1\right)eg_2\right)-D\left(m_2 + \left(1-\al_2\right)eg_1\right) \geq \\
			\left(1-\de^l\right) \left[T\left(s_2 + \al_2eg_1\right) + W\left(m_1 + \left(1-\al_1\right)eg_2\right) \right] \\
			+ \de^l \left(W\left(m_1 + \left(1-\al_1\right)eg_2\right)-D\left(m_2 + \left(1-\al_2\right)eg_1\right)\right)
			\label{eq:lowICex}
		\end{multline}

We've already assumed $e=1$ here as given in Lemma 2 for a concessions separating equilibrium 
	\begin{itemize}
		\item We \textit{should} assume $s_1,s_2,m_1,m_2$ to all be 1 to be consistent with the theory though. If we implement all that AND $e=1$, we get
			\begin{multline*}
					W\left(1 + \left(1-\al_1\right)g_2\right)-D\left(1 + \left(1-\al_2\right)g_1\right) \geq \left(1-\de^l\right) \left[T\left(1 + \al_2g_1\right) + W\left(1 + \left(1-\al_1\right)g_2\right) \right] \\
						+ \de^l \left(W\left(1 + \left(1-\al_1\right)g_2\right)-D\left(1 + \left(1-\al_2\right)g_1\right)\right)
			\end{multline*}
		\item We also know that there are no gifts exchanged for $X^l_{FF}$, so $g_1 = g_2 = 0$ on the LHS. On the RHS, gifts ARE exchanged because player 2 is a high type and the low type of player 1 lied. So we can set $g_1 = g_2 = g_M$
			\begin{multline*}
					W - D \geq \left(1-\de^l\right) \left[ T\left(1 + \al_2g_M\right) + W\left(1 + \left(1-\al_1\right)g_M\right) \right] \\
					+ \de^l \left(W\left(1 + \left(1-\al_1\right)g_M\right)-D\left(1 + \left(1-\al_2\right)g_M\right)\right)
			\end{multline*}
		\item The $\al_i$'s. Recall $\al_i$ represents the portion of the received gift that Player $i$ invests in civil society.
			\begin{itemize}
				\item Player 1 is low type (it is the low type IC constraint that I'm analyzing) and so $\al_1 = 0$.
				\item Player 2 is a high type in what's left of the RHS, so $\al_2 = 1$.
					\begin{itemize}
						\item The $(1-p)$ part of the RHS was combined with the LHS; it was for the low type of Player 2, so it matched with the LHS where there are no gifts since the mediator instructs no gifts if one low-type player tells the truth, which Player 2 does when we're looking at Player 1's IC constraint.
					\end{itemize}
				\item With $a_1 = 0$ and $a_2 = 1$, we have
					\begin{equation}
							W - D \geq \left(1-\de^l\right) \left[ T\left(1 + g_M\right) + W\left(1 + g_M\right) \right] + \de^l \left(W\left(1 + g_M\right)-D\right)
							\label{eqn:alphas}
					\end{equation}
			\end{itemize}
				
		\item Distributing, we have 
			\[
					W - D \geq T + Tg_M + W + Wg_M - \de^l T - \de^l Tg_M - \de^l W - \de^l Wg_M + \de^l W + \de^l Wg_M - \de^l D
			\]
				\item Canceling terms:
			\[
					- D \geq T + Tg_M + Wg_M - \de^l T - \de^l Tg_M - \de^l D
			\]
		\item Rearranging:
			\[
					0 \geq \left(1 - \de^l \right) T + \left(1 - \de^l \right) Tg_M + Wg_M + \left(1 - \de^l \right) D
			\]
			Assuming $g_M > 0$, this inequality can never hold since each of $T,W,D\geq0$ and $0 \leq \left(1 - \de^l \right) \leq 1$	
			\begin{itemize}
				\item What would it mean to have negative gifts?
			\end{itemize}					
		\end{itemize}


\subsection{What if gifts are inefficient?}
\label{sec:eless}
Here I attempt to use inefficient gifts to get the low type to truthfully reveal.\\

Equation \ref{eqn:IClow} is the general form of the IC constraint. In the next step in Section~\ref{sec:e1}, I let the gifts cancel out because we assume in that section that $e=1$. I need to go back to Equation \ref{eqn:IClow} here since $e$ is not assumed to be 1.
\[
	X_{FF}^l \geq pX_{FT}^l+(1-p)X_{FF}^l-pg_h+peg_h
\]
Leaving the two gift terms on the end intact, I simplify in the same way as in Section $\ref{sec:e1}$ to arrive at the correct analog to Equation~\ref{eq:lowICex}.
		\begin{multline*}
			W\left(m_1 + \left(1-\al_1\right)eg_2\right)-D\left(m_2 + \left(1-\al_2\right)eg_1\right) \geq \\
			\left(1-\de^l\right) \left[T\left(s_2 + \al_2eg_1\right) + W\left(m_1 + \left(1-\al_1\right)eg_2\right) \right] \\
			+ \de^l \left(W\left(m_1 + \left(1-\al_1\right)eg_2\right)-D\left(m_2 + \left(1-\al_2\right)eg_1\right)\right) -pg_h+peg_h
		\end{multline*}
	\begin{itemize}
		\item We assume $s_1,s_2,m_1,m_2$ to all be 1 to be consistent with Section 4.2
			\begin{multline*}
				W\left(1 + \left(1-\al_1\right)eg_2\right)-D\left(1 + \left(1-\al_2\right)eg_1\right) \geq \\
			\left(1-\de^l\right) \left[T\left(1 + \al_2eg_1\right) + W\left(1 + \left(1-\al_1\right)eg_2\right) \right] \\
			+ \de^l \left(W\left(1 + \left(1-\al_1\right)eg_2\right)-D\left(1 + \left(1-\al_2\right)eg_1\right)\right) -pg_h+peg_h
			\end{multline*}
		\item We know that there are no gifts exchanged for $X^l_{FF}$, so $g_1 = g_2 = 0$ on the LHS. On the RHS, gifts ARE exchanged because player 2 is a high type and the low type of player 1 lied. So we can set $g_1 = g_2 = g_M$:
			\begin{multline*}
				W\left(1 \right)-D\left(1 \right) \geq
			\left(1-\de^l\right) \left[T\left(1 + \al_2eg_M\right) + W\left(1 + \left(1-\al_1\right)eg_M\right) \right] \\
			+ \de^l \left(W\left(1 + \left(1-\al_1\right)eg_M\right)-D\left(1 + \left(1-\al_2\right)eg_M\right)\right) -pg_M+peg_M
			\end{multline*}
		\item With $a_1 = 0$ and $a_2 = 1$ (Recall the discussion above Equation~\ref{eqn:alphas}), we have
			\[
				W - D \geq \left(1-\de^l\right) \left[T\left(1 + eg_M\right) + W\left(1 + eg_M\right) \right] 
			+ \de^l \left(W\left(1 + eg_M\right) - D\right) -pg_M+peg_M
			\]
		\item It's time to start rearranging:
			\[
				pg_M\left(1 - e\right) \geq \left(1-\de^l\right) \left[T\left(1 + eg_M\right) + W\left(1 + eg_M\right) \right] 
			+ \de^l W\left(1 + eg_M\right) - \de^l D - W + D
			\]
			\[
				pg_M\left(1 - e\right) \geq \left(1-\de^l\right) \left[T\left(1 + eg_M\right) + W\left(1 + eg_M\right) \right] 
			+ \de^l W\left(1 + eg_M\right) + \left(1 - \de^l\right) D - W
			\]
			\begin{multline*}
			  	pg_M\left(1 - e\right) \geq T\left(1 + eg_M\right) + W\left(1 + eg_M\right) -\de^l T\left(1 + eg_M\right) -\de^l W\left(1 + eg_M\right) \\
			+ \de^l W\left(1 + eg_M\right) + \left(1 - \de^l\right) D - W
			\end{multline*}
			\[
				pg_M\left(1 - e\right) \geq T\left(1 + eg_M\right) + W\left(1 + eg_M\right) -\de^l T\left(1 + eg_M\right) + \left(1 - \de^l\right) D - W
			\]
			\[
				pg_M\left(1 - e\right) \geq T\left(1 + eg_M\right) + Weg_M -\de^l T\left(1 + eg_M\right) + \left(1 - \de^l\right) D
			\]
			\begin{equation}
				pg_M\left(1 - e\right) \geq \left(1 - \de^l\right) T\left(1 + eg_M\right) + Weg_M + \left(1 - \de^l\right) D
				\label{eq:examine}
			\end{equation}
	\end{itemize}

\subsubsection{Is it possible to pick e and g to satisfy low-type IC?}

\textbf{NOW, it's time to explore this equation and see if it's even possible to pick a gift high enough and an $e$ low enough to satisfy the low type's incentive constraint}. Continuing on from Equation~\ref{eq:examine} and simplifying so that $g_M = g$ and $\de^l = \de$,
\[
	pg\left(1 - e\right) \geq \left(T - \de T \right) \left(1 + eg\right) + Weg + \left(1 - \de \right) D
\]
	\begin{itemize}
		\item Distributing AGAIN
			\[
				pg	\left(1 - e\right) \geq T + Teg - \de T - \de T eg + Weg + \left(1 - \de \right) D
			\]
			\[
				pg	\left(1 - e\right) \geq \left(1 - \de \right) T + \left(1 - \de \right) D + Teg - \de T eg + Weg
			\]
			\begin{equation}
				pg - pge \geq \left(1 - \de \right) \left( T + D \right) + eg \left( T - \de T + W \right)
				\label{eq:split}
			\end{equation}
		\item From here we can solve either for the constraint in terms of $g$ or in terms of $e$
	\end{itemize}		
			
\subsubsection{Solve for g}			
			First solve for $g$ starting from Expression~\ref{eq:split}:
					$$pg - pge \geq \left(1 - \de \right) \left( T + D \right) + eg \left( T - \de T + W \right)$$
					$$pg - pge - eg \left( T - \de T + W \right)\geq \left(1 - \de \right) \left( T + D \right)$$
					\begin{equation}
						g \left(p - e \left( p + T - \de T + W \right) \right) \geq \left(1 - \de \right) \left( T + D \right)
						\label{eq:g1}
					\end{equation}
					\begin{equation}
						g \geq \frac{\left(1 - \de \right) \left( T + D \right)}{p - e \left( p + T - \de T + W \right)}
						\label{eq:g2}
					\end{equation}
						\begin{itemize}
							\item THIS ASSUMES {\color{red} $p - e \left( p + T - \de T + W \right)$ IS POSITIVE}
								\begin{itemize}
									\item Note that if we set $e=0$, this is a very simple constraint on $g$
										\begin{equation}
											g \geq \frac{\left(1 - \de \right) \left( T + D \right)}{p}
											\label{eq:g3}
										\end{equation}
								\end{itemize}
							\item $T+D$, $1 - \de $ non-negative by assumption $\Rightarrow$ RHS of Expression~\ref{eq:g1} non-negative
								\begin{itemize}
									\item If $\left(p - pe - e \left( T - \de T + W \right) \right)$ is negative, the only way the IC constraint in Expression~\ref{eq:g1} can hold is if $g<0$. AGAIN WITH THE NEGATIVE GIFTS.
								\end{itemize}
							\item To compare to the optimal gift size in the other sections:
								\begin{itemize}
									\item In Section 3 (Concessions Separating Equilibrium): $p\left(T+D\right)$
									\item Material value with $e=0$ same as in Section 3: $p\left(T+D\right)$
									\item Material value with $e=1$:
										\begin{equation}
											g \geq \frac{p(1-\de)(T + D)}{\left(1-\de \right) -p(1-\de)T +(1-p)D}
											\label{eq:g4}
										\end{equation}
									\item Under the assumption that payoffs are multiplied by something linear like $1 + \al_2 g_1$, there's no reason to choose any intermediate value of $e$. The mediator, of course, can specify an intermediate value.
									\item We can simplify to see that the minimum gift when $e=1$ is larger when
										\[
											(1-\de)pT > (1-p)D
										\]
										\color{red} BIG NEW NOTE: This condition only true if denominator above is positive
								\end{itemize}
						\end{itemize}
				
\subsubsection{Solve for e}
				Now solve for $e$ starting from Expression~\ref{eq:split}:
					$$pg - \left(1 - \de \right) \left( T + D \right) \geq peg + eg \left( T - \de T + W \right)$$
					\begin{equation}
						pg - \left(1 - \de \right) \left( T + D \right) \geq eg \left( p + T - \de T + W \right)
						\label{eq:g5}
					\end{equation}
						\begin{itemize}
							\item {\color{red} $g \left( p + T - \de T + W \right)$ is non-negative as long as $g$ is positive}: $p, 1-\de, T$ and $W$ are all assumed non-negative and $T+D > W$ so that $T$ is strictly positive. It can only possibly be zero if $p=0$ and $\de = 1$.
								\begin{itemize}
									\item For the IC constraint to hold then (with non-negative gifts), we must at least have {\color{red} $pg - \left(1 - \de \right) \left( T + D \right)$ non-negative, which requires $g$ to be set according to Expression~\ref{eq:g3}}. This only works if $e$ is set to zero; otherwise we need to look back at Expression~\ref{eq:g2}. This is just a weak condition that, if violated, means the IC constraint does not hold.
								\end{itemize}
							\item If {\color{red} $g \left( p + T - \de T + W \right)$ is positive}, we can divide through to get
								$$\frac{pg - \left(1 - \de \right) \left( T + D \right)}{g \left( p + T - \de T + W \right)} \geq e$$
						\end{itemize}
				To get a sense of the magnitudes, I will plug $g = p \left(T+D\right)$ from the $e=0$ CSE case (just because it's easier to work with than the optimal gift when $e=1$ into Expression~\ref{eq:g5}. First I divide through by $g$ {\color{blue}(this is okay if $g$ is assumed to be positive)}
						\begin{equation}
							p - \frac{\left(1 - \de \right) \left( T + D \right)}{g} \geq e \left( p + T - \de T + W \right)
							\label{eq:g6}
						\end{equation}
						$$p - \frac{\left(1 - \de \right) \left( T + D \right)}{p \left(T+D\right)} \geq e \left( p + T - \de T + W \right)$$
						$$p - \frac{1 - \de}{p } \geq e \left( p + T - \de T + W \right)$$
							\begin{itemize}
								\item For the LHS to be positive, we need $p^2 \geq 1 - \de$, e.g. if $p = 0.5$, we need $\de \geq .75$
							\end{itemize}
								
\subsubsection{Check with some parameterizations}								
								Parameterization that is in the paper now:
									\begin{itemize}
										\item Let $T = 10$, $W=8$ and $D = 5$
										\item Let $\de_h = .6$ and $\de_l = .3$. Assume $p = .2$. 
										\item Threshold for high types to separate when concessions do not have future material value is $\de = \frac{8}{10+5} = .5\overline{3}$. By Theorem 2, separation through concessions is possible.
										\item Minimum separating concession is $g^1 = \frac{p(1-d_l)(D+T)}{(1-d_l)(1-pT)+(1-p)D} = .\overline{63}$
										\item Minimum patience level required for separation is $\de_h^1 = D(1-p)+1-pT + \frac{p(W-(T+D))}{g^1} = .7\overline{9}.$ Because the specified $\de_h$ is not involved in any of these calculations, we can see that for any $\de_h \in [.5\overline{3},.7\overline{9})$, separating through concessions is possible when concessions have no material value but is not possible when they do have future material value.
										\item Minimum $e$ from Expression~\ref{eq:g6}:
											\[
											  0.2 - \frac{\left(1 - 0.3 \right) \left( 10 + 5 \right)}{g} \geq e \left( 0.2 + 10 - 0.3\cdot 10 + 8 \right)
											\]
											\[
											  0.2 - \frac{0.7 \left( 15 \right)}{g} \geq e \left( 11.2 \right)
											\]
											\[
											  \frac{0.2}{11.2} - \frac{0.7 \left( 15 \right)}{11.2g} \geq e 
											\]
											\[
											  0.01786 - \frac{9.375}{g} \geq e 
											\]
										To keep $e \geq 0$, we need $g \geq 525$ (this is REALLY big compared to $W$ for instance). It's also huge compared to the optimal gifts when $e=0$ of 3 and when $e=1$ of $0.\overline{63}$.
									\end{itemize}
								
							\vskip.2in	
							Now let's try the same parameterization with $p=0.9$.
									\begin{itemize}
										\item Minimum $e$ from Expression~\ref{eq:g6}:
											\[
											  0.9 - \frac{\left(1 - 0.3 \right) \left( 10 + 5 \right)}{g} \geq e \left( 0.9 + 10 - 0.3\cdot 10 + 8 \right)
											\]
											\[
											  0.9 - \frac{0.7 \left( 15 \right)}{g} \geq e \left( 11.9 \right)
											\]
											\[
											  \frac{0.9}{11.9} - \frac{0.7 \left( 15 \right)}{11.9g} \geq e 
											\]
											\[
											  0.07563 - \frac{8.8235}{g} \geq e 
											\]
										To keep $e \geq 0$, we need $g \geq 116.66$. This is still nearly an order of magnitude larger than $W$, and both the optimal gifts when $e=0$ of 13.5 and when $e=1$ of $\frac{p(1-d_l)(D+T)}{(1-d_l)(1-pT)+(1-p)D} = \frac{0.9(0.7)(15)}{0.7 \cdot (1-0.9)+(.1)5} = \frac{0.63\cdot15}{0.7 \cdot 0.1+.5} = 16.58$.
									\end{itemize}
							
\vskip.3in
``Parameterization 2'': goal was to make welfare higher under mediator; $We > W0$ and $e=1$ is not incentive compatible
\begin{itemize}
	\item \color{orange} T = 7, W = 8, D = 5, p = .9, $\de_l = .3$, $\de_h = .6$
	\item Minimum $e$ from Expression~\ref{eq:g6}
	  $$0.9 - \frac{\left(1 - 0.3 \right) \left( 7 + 5 \right)}{g} \geq e \left( 0.9 + 7 - 0.3 \cdot 7 + 8 \right)$$
		$$0.9 - \frac{8.4}{g} \geq e \left(18 \right)$$
		$$0.05 - \frac{0.466}{g} \geq e $$
	\item The highest welfare comes from $e=.06$, which implies $g = 116.67$ (see e-g calculations.R)
\end{itemize}
		
\vskip.2in
Parameterization 3: $W1 > W0$ and $g1 > g0$. Welfare under mediator is never positive. 
\begin{itemize}
	\item $T = 7, \ W = 8, \ D = 5, \ p = .5, \ \de_l = .25, \ \de_h = .6$
\end{itemize}

\vskip.2in
Parameterization 4: $W0 > W1 > We$ (mediator). Mediator optimally chooses e=0 anyway
\begin{itemize}
	\item $T = 7, \ W = 8, \ D = 5, \ p = .3, \ \de_l = .4, \ \de_h = .9$
\end{itemize}

\vskip.2in
Parameterization 5: $W1 > W0 > We$ (mediator). Mediator optimally chooses $e=0.03$
\begin{itemize}
	\item $T = 6, \ W = 8, \ D = 5, \ p = .5, \ \de_l = .2, \ \de_h = .9$
\end{itemize}

\vskip.2in
Parameterization 6: $We > W1 > W0$, $54 > 7.33 > 7$. Mediator optimally chooses $e=0.24$. Optimal gift is 50, effective gift is 12, vs. $g1 = .667$ and $g0 = 1$
\begin{itemize}
	\item $T = 1, \ W = 1, \ D = 1, \ p = .5, \ \de_l = .5, \ \de_h = 15/16=0.9375$
\end{itemize}
	
						
						\vskip.3in
						Welfare comparisons (always for high type)
							\begin{itemize}
								\item When $e=1$: $\frac{1}{1-\de_h}\left\{pT(1+g^*) + (1-p)(W-D(1+g^*))\right\} - \frac{p(1-d_l)(D+T)}{(1-d_l)(1-pT)+(1-p)D}$
									\begin{itemize}
										\item \color{red} Parameterization in the paper: welfare is 7.18; gift is 0.63
										\item \color{orange} Parameterization 2: $g^1$ is negative, creates negative upper bound for $\de_h$. Separating with concessions equilibrium can't be sustained.
									\end{itemize}
								\item When $e=0$: $\frac{1}{1-\de_h}\left\{pT + (1-p)(W-D)\right\} - p(T+D)$
									\begin{itemize}
										\item \color{red} Parameterization in the paper: welfare is 8; gift is 3
										\item \color{orange} Parameterization 2: welfare is 5.7; gift is 10.8
									\end{itemize}
								\item When $0<e<1$ under a mediator: $\frac{1}{1-\de_h}\left\{pT(1+eg_M) + (1-p)(W-D)\right\} - g_M$
									\begin{itemize}
										\item Optimal $g_M$: Expression~\ref{eq:g2} creates a menu of $(e,g)$ pairs that satisfy the constraint:
											\[
												g_M \geq \frac{\left(1 - \de \right) \left( T + D \right)}{p - e \left( p + T - \de T + W \right)}
											\]
										\item We're assuming $p - e \left( p + T - \de T + W \right) > 0$ (assuming gifts must be non-negative)
										\item Also $pg - \left(1 - \de \right) \left( T + D \right) \geq 0$ ($e$ version; this is WEAKER than the condition on $e$)
										\item \color{red} For the parameterization in the paper, welfare is negative for $e=0$ and $e=0.01$, the only values of $e$ that lead to non-negative gifts; gifts range from 52.5 to 218.75	
										\item \color{orange} Parameterization 2: welfare is 10.083 at optimal gift-efficiency $e=0.06$, $g = 116.67$, $eg = 7$
											\begin{itemize}
												\item Compared to $e=0$, welfare (gross of gift) goes up from 16.5 to 126.75. The entire difference is due to the positive effect of the material value of the gift (the full term is $\frac{1}{1-\de_h}pTeg$). The effective gift is $7$, which is much larger than the 1 in $\frac{1}{1-\de_h}pT\cdot 1$ that is there in both terms.
												\item The increase in the size (cost) of the gift from 10.8 to 116.67 is smaller than the increase in gross-of-gift welfare
											\end{itemize}
									\end{itemize}
							\end{itemize}

\subsubsection{Welfare under Mediator vs. e=0}
\label{sec:WM0}
Here I'll try to derive a general result for when the welfare under a mediator would be higher than under $e=0$ separating with concessions equilibrium. \\

When $0<e<1$ under a mediator, the optimal $g_M$ is the one that chooses the highest welfare from a menu of $(e,g)$ pairs:
$$\frac{1}{1-\de_h}\left\{pT(1+eg_M) + (1-p)(W-D)\right\} - \frac{\left(1 - \de_l \right) \left( T + D \right)}{p - e \left( p + T - \de_l T + W \right)}$$

When $e=0$, welfare is $$\frac{1}{1-\de_h}\left\{pT + (1-p)(W-D)\right\} - p(T+D)$$

We compare them to find when the welfare under the mediator is larger:
	\begin{multline*}
		\frac{1}{1-\de_h}\left\{pT(1+eg_M) + (1-p)(W-D)\right\} - \frac{\left(1 - \de_l \right) \left( T + D \right)}{p - e \left( p + T - \de_l T + W \right)} \geq \\
		\frac{1}{1-\de_h}\left\{pT + (1-p)(W-D)\right\} - p(T+D)$$
	\end{multline*}							
Eliminating common terms:
	\begin{equation*}
		\frac{1}{1-\de_h}\left\{pTeg_M \right\} - \frac{\left(1 - \de_l \right) \left( T + D \right)}{p - e \left( p + T - \de_l T + W \right)} \geq - p(T+D)
	\end{equation*}	
Rearranging:
	\begin{equation*}
		\frac{1}{1-\de_h}\left\{pTe - 1 \right\} \frac{\left(1 - \de_l \right) \left( T + D \right)}{p - e \left( p + T - \de_l T + W \right)} \geq - p(T+D)
	\end{equation*}	

Rearranging:
\begin{equation*}
		\frac{pTeg_M}{1-\de_h} \geq \frac{\left(1 - \de_l \right) \left( T + D \right)}{p - e \left( p + T - \de_l T + W \right)} - p(T+D)
	\end{equation*}	
This is what I implement in Mathematica Notebook ``We greater than W0.nb''
\begin{itemize}
	\item So far, I've been able to find an instance that satisfies this equation: e = 11/64, $\de_h = 15/16$, $\de_l = 0.5$, $p = 0.5$, T = 1, W = 1, D = 1
	\item This is ``Parameterization 6'' above and in R file ``e-g calculations.R''
\end{itemize}
			
\subsubsection{Welfare under Mediator vs. e=1}
\label{sec:WM1}
Here I'll try to derive a general result for when the welfare under a mediator would be higher than under $e=1$ separating with concessions equilibrium. \\

When $0<e<1$ under a mediator, the optimal $g_M$ is the one that chooses the highest welfare from a menu of $(e,g)$ pairs:
$$\frac{1}{1-\de_h}\left\{pT(1+eg_M) + (1-p)(W-D)\right\} - \frac{\left(1 - \de_l \right) \left( T + D \right)}{p - e \left( p + T - \de_l T + W \right)}$$

When $e=1$, welfare is $\frac{1}{1-\de_h}\left\{pT(1+g^1) + (1-p)(W-D(1+g^1))\right\} - \frac{p(1-d_l)(D+T)}{(1-d_l)(1-pT)+(1-p)D}$ \\

We compare them to find when the welfare under the mediator is larger:
	\begin{multline*}
		\frac{1}{1-\de_h}\left\{pT(1+eg_M) + (1-p)(W-D)\right\} - \frac{\left(1 - \de_l \right) \left( T + D \right)}{p - e \left( p + T - \de_l T + W \right)} \geq \\
		\frac{1}{1-\de_h}\left\{pT(1+g^1) + (1-p)(W-D(1+g^1))\right\} - \frac{p(1-d_l)(T+D)}{(1-d_l)(1-pT)+(1-p)D}
	\end{multline*}						
		
Eliminating common terms:
	\begin{multline*}
		\frac{1}{1-\de_h}\left\{pTeg_M \right\} - \frac{\left(1 - \de_l \right) \left( T + D \right)}{p - e \left( p + T - \de_l T + W \right)} \geq \\
		\frac{1}{1-\de_h}\left\{pTg^1 + (1-p)(- Dg^1))\right\} - \frac{p(1-d_l)(T+D)}{(1-d_l)(1-pT)+(1-p)D}
	\end{multline*}			
Rearranging:
	\begin{equation*}
		\frac{1}{1-\de_h}\left\{pTeg_M \right\} - g_M \geq \frac{1}{1-\de_h}\left\{pTg^1 + (1-p)(- Dg^1))\right\} - g^1
	\end{equation*}		
		
I've used Mathematica to find a parameterization under which this is true.
\begin{itemize}
	\item I implemented the constraint in Mathematica to make sure the gift under $e=1$ is positive, BUT I did not implement the constraint on $\de_h$ that makes it IC (although I have a check in ``e-g calculations.R''. Here is the constraint on $\de_h$:
		\[
		  \de_h \geq D + 1 - p(T+D) + \frac{p(W-(T+D))}{g^1}
		\]
\end{itemize}

			
\subsubsection{To do and Notes}		
	\color{ForestGreen} 
			\begin{itemize}
					\item NEXT STEP: check condition for sustainable $e=1$ but prefer mediator with $e < 1$ 
						\begin{itemize}
							\item I've found parameterizations where $e=1$ is preferred (P5)
							\item Try to find a parameterization that shows the reverse (counterexample to my hypothesis that if $e=1$ is sustainable, you never invoke mediator) or see if I can prove that it never happens
						\end{itemize}
					\item Can I derive an analytical expression for when the mediator enhances welfare? See Section~\ref{sec:WM0}
						\begin{itemize}
							\item This is likely (surely?) to be when efficient gifts can't be sustained as a CSE.
							\item If you can sustain $e=1$ with no mediator, would you ever want a mediator?
						\end{itemize}
				\item When $g^1$ turns negative, is that exactly when $e=1$ can't be sustained? If we can get positive $g^1$, not sure we'd need the mediator? It's not about $g^1$, but the $\de_h^1$ condition.
					\begin{itemize}
						\item When $g^1$ is negative, the constraint on $\de_h^1$ (see proof of Thm 3) becomes an upper bound. If it's negative, it means that no $\de_h$ exists to make separating possible.
						\item I need a formal result to see clearly what is going on here
					\end{itemize}
			\end{itemize}
	

\vskip.2in
\color{Black}Notes:
\begin{itemize}
	\item I think Mediator could instruct them to give $e<1$ gifts without having to disturb the rest of the theory. 
		\begin{itemize}
			\item \textbf{If I go with $e<1$, I need to come back to verify this.}
		\end{itemize}
\end{itemize}
			








\vskip.5in
\section{Develop result: mediator increases potential for peace}
\un{June 6, 2017}

\begin{itemize}
	\item i.e. want to show mediator makes (Trust, Trust) an equilibrium over a larger parameter space than without mediator
\end{itemize}

\vskip.2in
What do we know?
\begin{itemize}
	\item Theorem 4: Under some parameters, optimal concessions aren't made when there isn't trust
		\begin{itemize}
			\item This is still a separating equilibrium, but completely inefficient concessions are given. So welfare is lower because of inefficient concessions. 
			\item Where there is a possibility of mediation helping to achieve peace where it otherwise would not be attainable (not just improving welfare) is if this reduction in welfare due to the inefficient concessions means that for some parameters it's not worth separating so we don't get peace at all without the mediator.
				\begin{itemize}
					\item i.e. when welfare for high type from separating falls below that for pooling, separating is no longer an equilibrium (convo with Jean-Guillaume).
				\end{itemize}
		\end{itemize}
	\item Theorem 5: Mediator eliminates inefficient concessions
		\begin{itemize}
			\item Here, we're already in that parameter space where there \emph{are} inefficient concessions
		\end{itemize}
\end{itemize}

\vskip.4in
Need to start by assuming that $\de_h <$ threshold where separating with no concessions works.
\begin{itemize}
	\item \textbf{Check to see whether Theorem 4 proof has to change.}
\end{itemize}

\vskip.2in
When $e=1$ and no material value, $g_h^* = p(T+D)$ is the smallest concession for CSE. \\
When $e=1$ and concessions have material value, $g' = \frac{p(1-\de_l)(D+T)}{(1-\de_l)(1-pT) + (1-p)D}$ is the minimum separating concession [Theorem 3]
I claim in Theorem 5 that the minimum separating concession under mediation is $g^* = \frac{g'}{1-p}$.

\vskip.2in
With $\al = 1$,
\begin{itemize}
	\item SWOC (separating without concessions) when $\de_h \geq \frac{W}{(1-p)W + p(T+D)}$
	\item CSE (concessions separating equilibrium) when $\de_h \geq \frac{W}{T+D}$
\end{itemize}
With $\al \in [0,1)$ and $e=1$,
\begin{itemize}
	\item SWOC doesn't change because there are no concessions
	\item WTS there are parameters where can't get STC outcome because of $\al <1$
		\begin{itemize}
			\item $e < 1$ means some of these \textit{can} get to peace
			\item but some can't under \textit{any} value of $e$ [assuming $e$ chosen at same time as $g$]
		\end{itemize}
\end{itemize}
Remember that decision between equilibria is made not by what \textit{can} be achieved in terms of $\de$, but by best response, i.e. ``Does high type do better by not giving concession?''
\begin{itemize}
	\item Should check in my numerical examples
\end{itemize}

\vskip.4in
Intermediate step for Theorem 4
\begin{itemize}
	\item Low type IC constraint to solve for equilibrium high type gift
	\item Already having shown $g_l = 0$, can set $g_h = g$ for simplicity of notation
	\item Also, since all the discount factors are for the low type, I'll let $\de_l = \de$
	\item For the case where $e=1$:

\end{itemize}
IC constraint:
\[
  X_{FF}^l \geq p X_{FT}^l + (1-p) X_{FF}^l - g
\]
Expanded in basic terms (without efficiency issues)
\[
  \frac{W-D}{1-\de} \geq p \left[T + W + \frac{\de}{1-\de}\left(W-D\right)\right] + (1-p) \frac{W-D}{1-\de} - g
\]
Multiply through by $\left(1-\de \right)$
\[
  W-D \geq p\left(1-\de \right) \left[T + W\right] + p\de \left(W-D\right) + (1-p) \left[W-D\right] - \left(1-\de \right) g
\]
Now add complexity from Table 2
\begin{multline*}
  W(1+(1-\al_1)g_2)-D(1+(1-\al_2)g_1) \geq \\
	p\left(1-\de \right) \left[T(1+\al_2 g_1) + W(1+(1-\al_1)g_2)\right] + p\de \left(W(1+(1-\al_1)g_2)-D(1+(1-\al_2)g_1)\right) \\ + (1-p) \left[W(1+(1-\al_1)g_2)-D(1+(1-\al_2)g_1)\right] - \left(1-\de \right) g
\end{multline*}
Substitute in $\al_1 = 0$ everywhere since this is the IC for a low-type of player 1:
\begin{multline*}
  W(1+g_2) - D(1+(1-\al_2)g_1) \geq \\
	p\left(1-\de \right) \left[T(1+\al_2 g_1) + W(1+g_2) \right] + p\de \left(W(1+g_2) - D(1+(1-\al_2)g_1)\right) \\ + (1-p) \left[W(1+g_2) - D(1+(1-\al_2)g_1)\right] - \left(1-\de \right) g
\end{multline*}
Set $g_1 = 0$ on the LHS and $g_1 = g$ on the RHS:
\begin{multline*}
  W(1+g_2) - D \geq \\
	p\left(1-\de \right) \left[T(1+\al_2 g) + W(1+g_2) \right] + p\de \left(W(1+g_2) - D(1+(1-\al_2)g)\right) \\ + (1-p) \left[W(1+g_2) - D(1+(1-\al_2)g)\right] - \left(1-\de \right) g
\end{multline*}
Set $g_2 = 0$ and $\al_2 = 0$ wherever there is a $(1-p)$ and $g_2 = g$ and $\al_2 = 1$ wherever there is a $p$:
\begin{multline*}
  pW(1+g) + (1-p)W - D \geq \\
	p\left(1-\de \right) \left[T(1+g) + W(1+g) \right] + p\de \left(W(1+g) - D\right) \\ + (1-p) \left[W - D(1+g)\right] - \left(1-\de \right) g
\end{multline*}
Expand
\begin{multline*}
  pW + pWg + W - pW - D \geq \\
	\left(p-p\de \right) \left[T+Tg + W+Wg \right] + p\de \left(W+Wg - D\right) \\ + (1-p) \left[W - D -Dg\right] - \left(1-\de \right) g
\end{multline*}
Cancel some like terms and move $\left(1-\de \right) g$ to LHS
\begin{multline*}
  \left(1-\de \right) g + pWg + W - D \geq \\
	p\left[T+Tg + W+Wg \right] -p\de \left[T+Tg \right] - p\de D \\ + (1-p) \left[W - D -Dg\right] 
\end{multline*}
Do some more canceling and expanding
\begin{multline*}
  \left(1-\de \right) g + W - D \geq \\
	p\left[T+Tg + W\right] -p\de T -p\de Tg - p\de D \\ + W - D -Dg -pW +pD +pDg
\end{multline*}
\[
  \left(1-\de \right) g \geq 
	pT+pTg + pW -p\de T -p\de Tg - p\de D  -Dg -pW +pD +pDg
\]
\[
  \left(1-\de \right) g \geq pT+pTg -p\de T -p\de Tg - p\de D  -Dg +pD +pDg
\]
Now just rearrange to get all the $g$ terms on the left
\[
  \left(1-\de \right) g -pTg + p\de Tg + Dg - pDg\geq pT -p\de T - p\de D +pD
\]
\[
  \left[\left(1-\de \right) -p(1-\de)T +(1-p)D\right]g \geq p(1-\de)(T + D)
\]
\[
  g \geq \frac{p(1-\de)(T + D)}{\left(1-\de \right) -p(1-\de)T +(1-p)D}
\]
In comparing to the minimum separating gift when $e=0$, which is
\[
  g^* \geq \frac{p(1-\de)(T + D)}{\left(1-\de \right)} = p(T + D)
\]
We can simplify to see that the minimum gift when $e=1$ is larger when
\[
  (1-\de)pT > (1-p)D
\]

\vskip.5in
Next we get the threshold for $\de_h$ that is necessary for a concessions separating eqm to exist:
\begin{itemize}
	\item Use the high-type IC constraint:
		\[
		  p X^h_{TT} + (1-p) X^h_{FF} - g_h \geq X^h_{FF}
		\]
	\item We'll need to expand for material effects but then solve for both $e=0$ and $e=1$
	\item For $e=1$:
		\[
		  \frac{p}{1-\de_h}T(1+g) + \frac{1-p}{1-\de_h}(W - D(1+g)) - g_h \geq \frac{1}{1-\de_h}(W-D)
		\]
		\[
		  pT + pTg + (1-p)(W - D - Dg) - g_h(1-\de_h) \geq W-D
		\]
		\[
		  pT + pTg + W - D - Dg - pW + pD + pDg - g_h + g_h \de_h \geq W-D
		\]
		\[
		  pT + pTg - Dg - pW + pD + pDg - g_h + g_h \de_h \geq 0
		\]
		\[
		  pT + pTg - Dg - pW + pD + pDg - g_h + g_h \de_h \geq 0
		\]
		Note that all the $g$'s should be $g_h$'s in the five lines above:
		\[
		  g_h \de_h \geq - pT - pT g_h + D g_h + pW - pD - pD g_h + g_h
		\]
		\[
		  \de_h \geq \frac{p(W - D - T)}{g_h} + (1-p)D + 1 - pT 
		\]
	\item For $e=0$, you're back in the original case with no material value, i.e. enforceable for same $\de_h$ as no material value case
		\[
		  \frac{p}{1-\de_h}T + \frac{1-p}{1-\de_h}(W - D) - g_0 \geq \frac{1}{1-\de_h}(W-D)
		\]
		\[
		  pT + (1-p)(W - D) - (1-\de_h)g_0 \geq (W-D)
		\]
		\[
		  pT -p(W - D) + \de_h g_0 \geq g_0
		\]
		\[
		  \de_h g_0 \geq g_0 + p(W - D - T) 
		\]
		\[
		  \de_h \geq 1 + \frac{p(W - D - T)}{g_0}
		\]
		Optimal separating gift in this setting is $g_0 = p(D+T)$. Substituting this in, we have:
		\[
		  \de_h \geq 1 + \frac{p(W - D - T)}{p(D+T)} = 1 + \frac{pW - p(D - T)}{p(D+T)} = 1+\frac{pW}{p(D+T)}-1 = \frac{W}{D+T}
		\]
		
		\begin{itemize}
			\item But check: when $e=0$, also wipe out immediate monetary value? YES, in terms of benefit, but NOT cost, which is the only part that doesn't cancel out of IC constraints (same for high type and low type; benefit is on both sides; cost is on only one side)
				\begin{itemize}
					\item Not from utility standpoint, but from IC standpoint because direct benefit of gift cancels out
				\end{itemize}
		\end{itemize}
	\item But high type utility goes down
		\begin{itemize}
			\item Interestingly, if $e=0$, if high types would have separated without material value, they will here with material value that is destroyed (they don't do better in pooling). We see this by calculating the high type utility for the case where $e=0$, which is
				\[
				  \frac{1}{1-\de_h}\left[pT + (1-p)(W-D) \right] - g_0
				\]
				Substituting in the value of the $g_0$ (the optimal separating gift with $e=0$, which I have as $p(T+D)$ but I should check again to make sure it's correct for this situation) and the threshold value to separate when $e=0$ (i.e. $\de = \frac{W}{T+D}$), we have
				\[
				  \frac{T+D}{T+D-W}\left[pT + (1-p)(W-D) \right] - p(T+D)
				\]
				\[
				  \frac{T+D}{T+D-W}\left[p(T +D -W) + (W-D) \right] - p(T+D)
				\]
				\[
				  p(T+D) + \frac{T+D}{T+D-W}\left[W-D \right] - p(T+D)
				\]
				\[
				  \frac{T+D}{T+D-W}\left[W-D \right]
				\]
				This is exactly the payoff to the pooling equilibrium (that is, both types pool on fight with no gift giving) when $\de = \frac{W}{T+D}$, since $U_{pooling} = \frac{W-D}{1-\de}$.
				
				So $U_{g_0} = U_{pooling}$ at the cutoff for separating, and $U_{g_0}$ is larger for any discount factor higher than the cutoff. 
					\begin{itemize}
						\item This can be seen, for instance, by decomposing $U_{pooling}$ along the same lines as $U_{g_0}$ above into components for the cooperation vs. punishment and cost of the gift (even though that's not what's going on in pooling). The whole thing increases, including the cost of the gift portion for the pooling when $\de$ increases. But the cost of the gift does not scale up in $U_{g_0}$.
					\end{itemize}
		\end{itemize}
		We can also do this more generally. Take
			\[
			  U_{g_0} = \frac{1}{1-\de}\left(W-D\right)
			\]
			\[
				U_{pooling} = \frac{1}{1-\de_h}\left[pT + (1-p)(W-D) \right] - p(T+D) 
			\]
			\[
			  = pT + pD +\frac{\de}{1-\de}\left(pT + pD\right) + \frac{1}{1-\de} \left( W -D -pW\right) - pT -pD
			\]
			\[
			  = \frac{\de}{1-\de}\left(pT + pD\right) + \frac{1}{1-\de} \left( W -pW -D\right)
			\]
		Now compare the two, $U_{g_0} \lessgtr U_{pooling}$:
				\[
				  \frac{\de}{1-\de}\left(pT + pD\right) + \frac{1}{1-\de} \left( W -pW -D\right) \lessgtr \frac{1}{1-\de}\left(W-D\right)
				\]
				\[
				  \frac{\de}{1-\de}\left(pT + pD\right) - \frac{1}{1-\de} \left( pW \right) \lessgtr 0
				\]
				\[
				  \de\left(pT + pD\right) \lessgtr pW 
				\]
				\[
				  \de \lessgtr \frac{pW}{pT + pD} = \frac{W}{T+D}
				\]
			So that the utility from $U_{g_0}$ is greater exactly when the discount factor is above the threshold for concessions separating equilibrium to be possible without material value.
\end{itemize}

\vskip.2in
How does utility compare under $e=1$ and $e=0$? Compare the two, $U_{g_1} \lessgtr U_{g_0}$:
				\[
				  \frac{p}{1-\de_h}T(1+g) + \frac{1-p}{1-\de_h}(W - D(1+g)) - g_h \lessgtr \frac{1}{1-\de_h}\left[pT + (1-p)(W-D) \right] - g_0
				\]
				\[
				  pT(1+g_1) + \left(1-p\right)(W - D(1+g_1)) - \left(1-\de_h\right)g_1 \lessgtr pT + (1-p)(W-D) - \left(1-\de_h\right)g_0
				\]
				\[
				  pTg_1 + \left(1-p\right)(W - D -Dg_1) - \left(1-\de_h\right)g_1 \lessgtr (1-p)(W-D) - \left(1-\de_h\right)g_0
				\]
				\[
				  pTg_1 - \left(1-p\right)Dg_1 - \left(1-\de_h\right)g_1 \lessgtr - \left(1-\de_h\right)g_0
				\]
				\[
				  pTg_1 - \left(1-p\right)Dg_1  \lessgtr \left(1-\de_h\right)g_1 - \left(1-\de_h\right)g_0
				\]
				\begin{equation}
				  pT - \left(1-p\right)D  \lessgtr \frac{\left(1-\de_h\right)\left(g_1 - g_0 \right)}{g_1}
					\label{eq:util}
				\end{equation}

We know (from above) that the minimum gift when $e=1$ is larger than the minimum gift when $e=0$ when
\[
  (1-\de)pT > (1-p)D
\]

\vskip.2in
Let's case this out.
\begin{enumerate}
	\item Assume $g_0 = g_1$; then $(1-\de)pT = (1-p)D$. The right hand side of \ref{eq:util} is 0 and the left hand side is
		\[
			pT - (1-\de)pT = pT - pT + \de pT = \de pT 
		\]
		And utility under $e=1$ is higher.
	\item When $g_0 > g_1$; then $(1-\de)pT < (1-p)D$. The right hand side of \ref{eq:util} is negative and the left hand side is larger (more positive) than in case 1, so utility under $e=1$ is also higher in this case.
	\item When $g_1 > g_0$; then $(1-\de)pT > (1-p)D$. The right hand side of \ref{eq:util} becomes positive and the left hand side becomes smaller (less positive, and possibly negative) than in case 1, so utility under $e=1$ is higher when $g_1$ is only a little larger than $g_0$, but eventually as $g_1$ gets much larger than $g_0$, utility under $e=0$ becomes larger than utility under $e=1$.
\end{enumerate}


\newpage
\section{To-do list after Venice CesIfo 06/15/17}
\begin{itemize}
	\item Jim Fearon: maybe get rid of repeated game and just parameterize a one-shot game (doens't work because concessions required for separation are too large; require future to recoup cost)
		\begin{itemize}
			\item he has a paper about concessions being used against you; he also gave me the reference for another one but I've forgotten it
		\end{itemize}

\vskip.3in
Notes from CPEG, Oct. 25, 2019 presentation:
\begin{itemize}
	\item Arnaud
		\begin{itemize}
			\item $\de$ are not exogenous (look at Brexit)
			\item Proposes a one-shot game to replace the repeated game. Simple bayesian game with difference preferences depening on type.
		\end{itemize}
	\item Raphael Godefroy
		\begin{itemize}
			\item Can $\de_i$ be a function of $g$? That is, likelihood of defection decreases as concession gets better
		\end{itemize}
\end{itemize}

\newpage
\section{Review 1 from EJPE (Email from Toke Aidt, August 19, 2018)}

While clearly these are interesting issues, I have difficulties reading this paper. Everything is presented in a very unclear manner. Game-theoretical concepts are invoked, but typically not correctly. For instance, an equilibrium is a strategy profile (one strategy for each type of each player). Hence it does not make sense to say, as the authors do in Theorem 1 on page 10, that something is an equilibrium strategy for one type without also specifying what others do. As another example, on page 18, the authors say that “The revelation principle is used to attain truthful self-identification.” But in reality, the revelation principle is just an argument for why it is without loss of generality to restrict attention to direct mechanisms. The truthfulness of messages is guaranteed by requiring the mechanism to be incentive compatible, something the authors do not address. \\

Another thing. It is not considered a good idea to introduce new assumptions, which have not been mentioned before, in the proof of a proposition, as happens in the alleged proof of Theorem 4 on page 27.

\end{itemize}
\end{document}